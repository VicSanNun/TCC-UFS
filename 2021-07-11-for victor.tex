
\documentclass[a4paper,12pt]{article}
%%%%%%%%%%%%%%%%%%%%%%%%%%%%%%%%%%%%%%%%%%%%%%%%%%%%%%%%%%%%%%%%%%%%%%%%%%%%%%%%%%%%%%%%%%%%%%%%%%%%%%%%%%%%%%%%%%%%%%%%%%%%%%%%%%%%%%%%%%%%%%%%%%%%%%%%%%%%%%%%%%%%%%%%%%%%%%%%%%%%%%%%%%%%%%%%%%%%%%%%%%%%%%%%%%%%%%%%%%%%%%%%%%%%%%%%%%%%%%%%%%%%%%%%%%%%
\usepackage{amssymb}
\usepackage{amsfonts}
\usepackage{graphicx}
\usepackage{amsmath}

\setcounter{MaxMatrixCols}{10}
%TCIDATA{OutputFilter=LATEX.DLL}
%TCIDATA{Version=5.50.0.2953}
%TCIDATA{<META NAME="SaveForMode" CONTENT="1">}
%TCIDATA{BibliographyScheme=Manual}
%TCIDATA{Created=Sat Oct 14 01:01:49 2006}
%TCIDATA{LastRevised=Tuesday, July 13, 2021 16:52:12}
%TCIDATA{<META NAME="GraphicsSave" CONTENT="32">}
%TCIDATA{<META NAME="DocumentShell" CONTENT="Journal Articles\Standard LaTeX Article">}
%TCIDATA{CSTFile=LaTeX article (bright).cst}

\textheight=26.5cm
\textwidth=18cm
\oddsidemargin=-1.0cm
\evensidemargin=-1.0cm
\topmargin=-2.5cm
\def\baselinestretch{1.6}

\begin{document}


;

\section{Laplaciano na dim N+1}

;

We use the (N+1)-dimensional ellipsoidal coordinates%
\begin{eqnarray*}
x_{N+1} &=&a\sinh \chi \cos \theta _{N} \\
x_{N} &=&a\cosh \chi \sin \theta _{N}\cos \theta _{N-1} \\
x_{N-1} &=&a\cosh \chi \sin \theta _{N}\sin \theta _{N-1}\cos \theta _{N-2}
\\
&&... \\
x_{3} &=&a\cosh \chi \sin \theta _{N}...\sin \theta _{3}\cos \theta _{2} \\
x_{1} &=&a\cosh \chi \sin \theta _{N}...\sin \theta _{3}\sin \theta _{2}\cos
\theta _{1} \\
x_{2} &=&a\cosh \chi \sin \theta _{N}...\sin \theta _{3}\sin \theta _{2}\sin
\theta _{1} \\
0 &\leq &\chi <\infty ,\ 0\leq \theta _{i}\leq \pi ,\ 2\leq i\leq N,\ 0\leq
\theta _{1}<2\pi
\end{eqnarray*}%
they describe a family of N-hiperboloids and N-ellipsoids:%
\begin{equation*}
\frac{1}{a^{2}\sin ^{2}\theta _{N}}\sum_{i=1}^{N}x_{i}^{2}-\frac{1}{%
a^{2}\cos ^{2}\theta _{N}}x_{N+1}^{2}=1,\ \frac{1}{a^{2}\cosh ^{2}\chi }%
\sum_{i=1}^{N}x_{i}^{2}+\frac{1}{a^{2}\sinh ^{2}\chi }x_{N+1}^{2}=1
\end{equation*}

The Laplacian in the ellipsoidal coordinates in N+1 dimensions is given by
the expression:%
\begin{equation*}
\nabla ^{2}u=\frac{1}{a^{2}\left( \sinh ^{2}\chi +\cos ^{2}\theta
_{N}\right) }\nabla _{\theta _{N},\chi }^{2}u+\frac{1}{a^{2}\cosh ^{2}\chi
\sin ^{2}\theta _{N}}\nabla _{S^{N-1}}^{2}u
\end{equation*}%
where $\nabla _{S^{N-1}}^{2}$ is the Laplacian on the (N-1)-dimensional unit
sphere and%
\begin{equation*}
\nabla _{\theta _{N},\chi }^{2}=\left( \frac{1}{\sin ^{N-1}\theta _{N}}\frac{%
\partial }{\partial \theta _{N}}\left( \sin ^{N-1}\theta _{N}\frac{\partial u%
}{\partial \theta _{N}}\right) +\frac{1}{\cosh ^{N-1}\chi }\frac{\partial }{%
\partial \chi }\left( \cosh ^{N-1}\chi \frac{\partial u}{\partial \chi }%
\right) \right) 
\end{equation*}%
\begin{equation*}
\end{equation*}%
;

\section{proximo passo}

;

separa\c{c}\~{a}o de vari\'{a}veis, ainda para 4d

aplicar o troco de vari\'{a}veis 

\begin{equation*}
\xi =\cos \theta _{3},\ \zeta =\sinh \chi ,\ \theta _{1},\theta _{2}
\end{equation*}%
\begin{equation*}
-1\leq \xi \leq 1,\ 0\leq \zeta <\infty ,\ 0\leq \theta _{1}<2\pi ,\ 0\leq
\theta _{2},\theta _{3}<\pi 
\end{equation*}%
e separar a parte com $\left( \theta _{1},\theta _{2}\right) $ e a parte com 
$\left( \theta _{3},\chi \right) $%
\begin{equation*}
\end{equation*}%
;

\end{document}
