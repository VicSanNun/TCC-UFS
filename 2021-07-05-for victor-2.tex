
\documentclass[a4paper,12pt]{article}
%%%%%%%%%%%%%%%%%%%%%%%%%%%%%%%%%%%%%%%%%%%%%%%%%%%%%%%%%%%%%%%%%%%%%%%%%%%%%%%%%%%%%%%%%%%%%%%%%%%%%%%%%%%%%%%%%%%%%%%%%%%%%%%%%%%%%%%%%%%%%%%%%%%%%%%%%%%%%%%%%%%%%%%%%%%%%%%%%%%%%%%%%%%%%%%%%%%%%%%%%%%%%%%%%%%%%%%%%%%%%%%%%%%%%%%%%%%%%%%%%%%%%%%%%%%%
\usepackage{amssymb}
\usepackage{amsfonts}
\usepackage{graphicx}
\usepackage{amsmath}

\setcounter{MaxMatrixCols}{10}
%TCIDATA{OutputFilter=LATEX.DLL}
%TCIDATA{Version=5.50.0.2953}
%TCIDATA{<META NAME="SaveForMode" CONTENT="1">}
%TCIDATA{BibliographyScheme=Manual}
%TCIDATA{Created=Sat Oct 14 01:01:49 2006}
%TCIDATA{LastRevised=Monday, July 05, 2021 16:57:35}
%TCIDATA{<META NAME="GraphicsSave" CONTENT="32">}
%TCIDATA{<META NAME="DocumentShell" CONTENT="Journal Articles\Standard LaTeX Article">}
%TCIDATA{CSTFile=LaTeX article (bright).cst}

\textheight=26.5cm
\textwidth=18cm
\oddsidemargin=-1.0cm
\evensidemargin=-1.0cm
\topmargin=-2.5cm
\def\baselinestretch{1.6}
\input{tcilatex}
\begin{document}


;

corre\c{c}\~{a}o

para preencher todo o espa\c{c}o deve ser assim:

o sistema de hiperboloides de uma folha%
\begin{eqnarray*}
x &=&r\cosh v\cos \theta  \\
y &=&r\cosh v\sin \theta  \\
z &=&r\sinh v
\end{eqnarray*}%
\begin{equation*}
0<r<\infty ,\ 0\leq \theta <2\pi ,\ -\infty <v<\infty 
\end{equation*}%
que obedecem%
\begin{equation*}
x^{2}+y^{2}-z^{2}=r^{2}
\end{equation*}%
mais o sistema de hiperboloides de duas folhas%
\begin{eqnarray*}
x &=&r\sinh v\cos \theta  \\
y &=&r\sinh v\sin \theta  \\
z &=&\pm r\cosh v
\end{eqnarray*}%
\begin{equation*}
0<r<\infty ,\ 0\leq \theta <2\pi ,\ -\infty <v<\infty 
\end{equation*}%
que obedecem%
\begin{equation*}
x^{2}+y^{2}-z^{2}=-r^{2}
\end{equation*}%
e ainda uma superf\'{\i}cie c\'{o}nica (para completar o espe\c{c}o):%
\begin{eqnarray*}
x &=&\sinh v\cos \theta  \\
y &=&\sinh v\sin \theta  \\
z &=&\sinh v
\end{eqnarray*}%
\begin{equation*}
0\leq \theta <2\pi ,\ -\infty <v<\infty 
\end{equation*}%
que obedecem%
\begin{equation*}
x^{2}+y^{2}-z^{2}=0
\end{equation*}%
\begin{equation*}
\end{equation*}%
;

\end{document}
