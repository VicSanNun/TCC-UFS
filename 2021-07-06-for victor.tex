
\documentclass[a4paper,12pt]{article}
%%%%%%%%%%%%%%%%%%%%%%%%%%%%%%%%%%%%%%%%%%%%%%%%%%%%%%%%%%%%%%%%%%%%%%%%%%%%%%%%%%%%%%%%%%%%%%%%%%%%%%%%%%%%%%%%%%%%%%%%%%%%%%%%%%%%%%%%%%%%%%%%%%%%%%%%%%%%%%%%%%%%%%%%%%%%%%%%%%%%%%%%%%%%%%%%%%%%%%%%%%%%%%%%%%%%%%%%%%%%%%%%%%%%%%%%%%%%%%%%%%%%%%%%%%%%
\usepackage{amssymb}
\usepackage{amsfonts}
\usepackage{graphicx}
\usepackage{amsmath}

\setcounter{MaxMatrixCols}{10}
%TCIDATA{OutputFilter=LATEX.DLL}
%TCIDATA{Version=5.50.0.2953}
%TCIDATA{<META NAME="SaveForMode" CONTENT="1">}
%TCIDATA{BibliographyScheme=Manual}
%TCIDATA{Created=Sat Oct 14 01:01:49 2006}
%TCIDATA{LastRevised=Tuesday, July 06, 2021 22:50:12}
%TCIDATA{<META NAME="GraphicsSave" CONTENT="32">}
%TCIDATA{<META NAME="DocumentShell" CONTENT="Journal Articles\Standard LaTeX Article">}
%TCIDATA{CSTFile=LaTeX article (bright).cst}

\textheight=26.5cm
\textwidth=18cm
\oddsidemargin=-1.0cm
\evensidemargin=-1.0cm
\topmargin=-2.5cm
\def\baselinestretch{1.6}
\begin{document}


;

\section{Spheroidal coordinates}

;

vamos combinar de nota\c{c}\~{o}es. usam essas mesmas!

$a$ \'{e} uma constante; deixamos por enquanto, pode ser \'{u}til no futuro

prolate spheroidal coordinates - two-sheet hypeboloids%
\begin{eqnarray*}
x &=&a\sinh v\sin \theta \cos \varphi  \\
y &=&a\sinh v\sin \theta \sin \varphi  \\
z &=&a\cosh v\cos \theta 
\end{eqnarray*}%
\begin{equation*}
0\leq v<\infty ,\ 0\leq \theta \leq \pi ,\ 0\leq \varphi <2\pi 
\end{equation*}%
;

oblate spheroidal coordinates - one-sheet hypeboloids - o nosso caso atual%
\begin{eqnarray*}
x &=&a\cosh v\cos \theta \cos \varphi  \\
y &=&a\cosh v\cos \theta \sin \varphi  \\
z &=&a\sinh v\sin \theta 
\end{eqnarray*}%
\begin{equation*}
0\leq v<\infty ,\ -\pi /2\leq \theta \leq \pi /2,\ 0\leq \varphi <2\pi 
\end{equation*}%
;

\end{document}
