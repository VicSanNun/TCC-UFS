\documentclass[12pt, a4papper]{article}
\usepackage[utf8]{inputenc}
\usepackage[brazil]{babel}
\usepackage{amsmath}
\usepackage{amsfonts}
\usepackage{amssymb}
\usepackage{verbatim}
\usepackage[T1]{fontenc}
\usepackage{graphicx}
\usepackage{physics}
\usepackage{siunitx}
\usepackage[top=1cm, bottom=1cm, left=1cm, right=1cm]{geometry}
\usepackage{makeidx}
\usepackage[utf8]{inputenc}
\newtheorem{defi}{Definição}[subsection]
\newtheorem{teor}{Teorema}[subsection]
\newtheorem{demo}{Demonstração}[subsection]
\newtheorem{nota}{Notação}[subsection]
\newtheorem{coro}{Corolário}[subsection]

\begin{document}

\section{Laplaciano hiperboloide de 2 folhas (3-dim)}

Dada a seguinte parametrização:

\begin{eqnarray*}
x_{3} & = & a\cosh\chi\cos\theta_{2}\\
x_{2} & = & a\sinh\chi\sin\theta_{2}\cos\theta_{1}\\
x_{1} & = & a\sinh\chi\sin\theta_{2}\sin\theta_{1}\\
0 & \leq & \chi<\infty,\ 0\leq\theta_{2}\leq\pi,\ 0\leq\theta_{1}<2\pi
\end{eqnarray*}

com

\[
q_{1}=\theta_{1},\ q_{2}=\theta_{2},\ q_{3}=\chi
\]

Utiliza-se o Operador de Laplace-Beltrami para encontrar o Laplaciano:

\[
\Delta u=\frac{1}{\sqrt{\left\vert g\right\vert }}\frac{\partial}{\partial q_{i}}\left(\sqrt{\left\vert g\right\vert }g^{ij}\frac{\partial u}{\partial q_{j}}\right)
\]

De acordo com Arfken(2017), pode-se determinar o tensor métrico $g_{ij}$

\[
G=(g_{ij})=\underset{k}{\sum}\frac{\partial x_{k}}{\partial q_{_{^{i}}}}\frac{\partial x_{^{_{k}}}}{\partial q_{_{^{j}}}}
\]

Na dada parametrização, forma-se um \textbf{sistema de coordenadas
ortogonal} de modo que, segundo Riley(2006), os elementos do tensor
métrico podem ser dados por:
\[
g_{ij}=\begin{cases}
h_{i}^{2} & ,i=j\\
0 & ,i\neq j
\end{cases}
\]

Onde $h_{i}^{2}=\sum_{j}\left[\left(\frac{\partial x_{j}}{\partial q_{i}}\right)^{2}\right]$.
Ou seja, $h_{i}$ é associado a $q_{i}$. Como nesse caso $q_{1}=\theta_{1},\,q_{2}=\theta_{2},\,q_{3}=\chi$
então tem-se $h_{1}=h_{\theta_{1}}$, $h_{2}=h_{\theta_{2}}$ e $h_{3}=\chi$.

Ou seja, a matriz $G=(g_{ij})$ é diagonal com os elementos $h_{1}^{2}$,
$h_{2}^{2}$ e $h_{3}^{2}$. Em outras palavras $G=diag(h_{1}^{2},\,h_{2}^{2},\,h_{3}^{2})$.

\[
G=(g_{ij})=\left(\begin{array}{ccc}
h_{1}^{2} & 0 & 0\\
0 & h_{2}^{2} & 0\\
0 & 0 & h_{3}^{2}
\end{array}\right)
\]

Ainda de acordo com Riley(2006), o tensor métrico inverso, para um
sistema ortogonal de coordenadas, pode ser dado por:

\[
G^{-1}=(g^{ij})=\begin{cases}
1/h_{i}^{2} & ,i=j\\
0 & ,i\neq j
\end{cases}
\]

\[
G^{-1}=(g^{ij})=\left(\begin{array}{ccc}
\frac{1}{h_{1}^{2}} & 0 & 0\\
0 & \frac{1}{h_{2}^{2}} & 0\\
0 & 0 & \frac{1}{h_{3}^{2}}
\end{array}\right)
\]

E, por fim, $g=det(g_{ij})=h_{1}^{2}h_{2}^{2}h_{3}^{2}$ . Consequentemente,
$\sqrt{|g|}=h_{1}h_{2}h_{3}$.

Os coeficientes métricos são encontrados da seguinte forma:

\[
h_{1}^{2}=h_{\theta_{1}}^{2}=\left(\frac{\partial x_{1}}{\partial\theta_{1}}\right)^{2}+\left(\frac{\partial x_{2}}{\partial\theta_{1}}\right)^{2}+\left(\frac{\partial x_{3}}{\partial\theta_{1}}\right)^{2}
\]

\[
h_{2}^{2}=h_{\theta_{2}}^{2}=\left(\frac{\partial x_{1}}{\partial\theta_{2}}\right)^{2}+\left(\frac{\partial x_{2}}{\partial\theta_{2}}\right)^{2}+\left(\frac{\partial x_{3}}{\partial\theta_{2}}\right)^{2}
\]

\[
h_{3}^{2}=h_{\chi}^{2}=\left(\frac{\partial x_{1}}{\partial\chi}\right)^{2}+\left(\frac{\partial x_{2}}{\partial\chi}\right)^{2}+\left(\frac{\partial x_{3}}{\partial\chi}\right)^{2}
\]

Calcula-se tais coeficientes:
\begin{align*}
h_{1}^{2} & =h_{\theta_{1}}^{2}=\left(\frac{\partial x_{1}}{\partial\theta_{1}}\right)^{2}+\left(\frac{\partial x_{2}}{\partial\theta_{1}}\right)^{2}=a^{2}sinh^{2}(\chi)cos^{2}(\theta_{1})sin^{2}(\theta_{2})+a^{2}sinh^{2}(\chi)sin^{2}(\theta_{1})sin^{2}(\theta_{2})
\end{align*}

\[
=a{{}^2}sinh^{2}(\chi)sin^{2}(\theta_{2})\left[cos^{2}(\theta_{1})+sin^{2}(\theta_{1})\right]
\]

\[
\boldsymbol{h_{1}^{2}=h_{\theta_{1}}^{2}=a^{2}sinh^{2}(\chi)sin^{2}(\theta_{2})}
\]
\\
\\
\[
h_{2}^{2}=h_{\theta_{2}}^{2}=\left(\frac{\partial x_{1}}{\partial\theta_{2}}\right)^{2}+\left(\frac{\partial x_{2}}{\partial\theta_{2}}\right)^{2}+\left(\frac{\partial x_{3}}{\partial\theta_{2}}\right)^{2}=a^{2}sinh^{2}(\chi)sin^{2}(\theta_{1})cos^{2}(\theta_{2})
\]

\[+a^{2}sinh^{2}(\chi)cos^{2}(\theta_{1})cos^{2}(\theta_{2})+a^{2}cosh^{2}(\chi)sin^{2}(\theta_{2}) \]

\[
=a^{2}sinh^{2}(\chi)cos^{2}(\theta_{2})\left[sin^{2}(\theta_{1})+cos^{2}(\theta_{1})\right]+a^{2}cosh^{2}(\chi)sin^{2}(\theta_{2})
\]

\[
=a^{2}sinh^{2}(\chi)cos^{2}(\theta_{2})+a^{2}cosh^{2}(\chi)sin^{2}(\theta_{2})
\]

\[
\boldsymbol{h_{2}^{2}=h_{\theta_{2}}^{2}=a^{2}\left[sinh^{2}(\chi)cos^{2}(\theta_{2})+cosh^{2}(\chi)sin^{2}(\theta_{2})\right]}
\]
\\
\\
\[
h_{3}^{2}=h_{\chi}^{2}=\left(\frac{\partial x_{1}}{\partial\chi}\right)^{2}+\left(\frac{\partial x_{2}}{\partial\chi}\right)^{2}+\left(\frac{\partial x_{3}}{\partial\chi}\right)^{2}=a^{2}cosh^{2}(\chi)sin^{2}(\theta_{1})sin^{2}(\theta_{2})+a^{2}cosh^{2}(\chi)cos^{2}(\theta_{1})sin^{2}(\theta_{2})
\]

\[ +a^{2}sinh^{2}(\chi)cos^{2}(\theta_{2})\]

\[
=a^{2}cosh^{2}(\chi)sin^{2}(\theta_{2})\left[sin^{2}(\theta_{1})+cos^{2}(\theta_{1})\right]+a^{2}sinh^{2}(\chi)cos^{2}(\theta_{2})
\]

\[
=a^{2}cosh^{2}(\chi)sin^{2}(\theta_{2})+a^{2}sinh^{2}(\chi)cos^{2}(\theta_{2})
\]

\[
\boldsymbol{h_{3}^{2}=h_{\chi}^{2}=a^{2}\left[cosh^{2}(\chi)sin^{2}(\theta_{2})+sinh^{2}(\chi)cos^{2}(\theta_{2})\right]}
\]
\\
\\
Observa-se que $h_{2}^{2}=h_{\theta_{2}}^{2}=h_{3}^{2}=h_{\chi}^{2}$
então $\sqrt{|g|}=h_{1}h_{2}^{2}$. Agora é possível construir a matriz
$\sqrt{|g|}g^{ij}$:

\[
\sqrt{|g|}g^{ij}=h_{1}h_{2}^{2}\left(\begin{array}{ccc}
\frac{1}{h_{1}^{2}} & 0 & 0\\
0 & \frac{1}{h_{2}^{2}} & 0\\
0 & 0 & \frac{1}{h_{3}^{2}}
\end{array}\right)=\left(\begin{array}{ccc}
\frac{h_{2}^{2}}{h_{1}} & 0 & 0\\
0 & h_{1} & 0\\
0 & 0 & h_{1}
\end{array}\right)
\]

Aplica-se os resultados no Operador de Laplace-Beltrami e chega-se
em:

\[
\Delta u=\frac{1}{\sqrt{\left\vert g\right\vert }}\left[\frac{\partial}{\partial q_{1}}\left(\sqrt{\left\vert g\right\vert }g^{11}\frac{\partial u}{\partial q_{1}}\right)+\frac{\partial}{\partial q_{2}}\left(\sqrt{\left\vert g\right\vert }g^{22}\frac{\partial u}{\partial q_{2}}\right)+\frac{\partial}{\partial q_{3}}\left(\sqrt{\left\vert g\right\vert }g^{33}\frac{\partial u}{\partial q_{3}}\right)\right]
\]
\\
\\
\begin{align*}
\Delta u & =\frac{1}{h_{1}h_{2}^{2}}\left[\frac{\partial}{\partial\theta_{1}}\left(\frac{h_{2}^{2}}{h_{1}}\frac{\partial u}{\partial\theta_{1}}\right)+\frac{\partial}{\partial\theta_{2}}\left(h_{1}\frac{\partial u}{\partial\theta_{2}}\right)+\frac{\partial}{\partial\chi}\left(h_{1}\frac{\partial u}{\partial\chi}\right)\right]
\end{align*}
\\
\\
\[
\Delta u=\frac{1}{h_{1}h_{2}^{2}}\left[\frac{\partial}{\partial\theta_{1}}\left(\frac{a\left[sinh^{2}(\chi)cos^{2}(\theta_{2})+cosh^{2}(\chi)sin^{2}(\theta_{2})\right]}{sinh(\chi)sin(\theta_{2})}\frac{\partial u}{\partial\theta_{1}}\right) \right.
\]

\[ 
\left. +\frac{\partial}{\partial\theta_{2}}\left(asinh(\chi)sin(\theta_{2})\frac{\partial u}{\partial\theta_{2}}\right)+\frac{\partial}{\partial\chi}\left(asinh(\chi)sin(\theta_{2})\frac{\partial u}{\partial\chi}\right)\right]
\]
\\
\\
\[
\Delta u=\frac{1}{h_{1}h_{2}^{2}}\left[\frac{a\left[sinh^{2}(\chi)cos^{2}(\theta_{2})+cosh^{2}(\chi)sin^{2}(\theta_{2})\right]}{sinh(\chi)sin(\theta_{2})}\frac{\partial{{}^2}u}{\partial\theta{{}^2}_{1}}+\frac{\partial}{\partial\theta_{2}}\left(asinh(\chi)sin(\theta_{2})\frac{\partial u}{\partial\theta_{2}}\right) \right.
\]

\[ \left. +\frac{\partial}{\partial\chi}\left(asinh(\chi)sin(\theta_{2})\frac{\partial u}{\partial\chi}\right)\right] \]
\\
\\
Sabendo que $\frac{1}{h_{1}h_{2}^{2}}=\frac{1}{a^{3}sinh(\chi)sin(\theta_{2})\left[sinh^{2}(\chi)cos^{2}(\theta_{2})+cosh^{2}(\chi)sin^{2}(\theta_{2})\right]}$,
o Laplaciano fica:
\\
\\
\[
\Delta u=\frac{1}{h_{1}h_{2}^{2}}\left[\frac{\partial}{\partial\theta_{2}}\left(asinh(\chi)sin(\theta_{2})\frac{\partial u}{\partial\theta_{2}}\right)+\frac{\partial}{\partial\chi}\left(asinh(\chi)sin(\theta_{2})\frac{\partial u}{\partial\chi}\right)\right]+\frac{1}{a^{2}sinh^{2}(\chi)sin^{2}(\theta_{2})}\frac{\partial{{}^2}u}{\partial\theta{{}^2}_{1}}
\]

\[
\boldsymbol{\Delta u=\frac{1}{a^{3}sinh(\chi)sin(\theta_{2})\left[sinh^{2}(\chi)cos^{2}(\theta_{2})+cosh^{2}(\chi)sin^{2}(\theta_{2})\right]}\left[\frac{\partial}{\partial\theta_{2}}\left(asinh(\chi)sin(\theta_{2})\frac{\partial u}{\partial\theta_{2}}\right) \right.}
\]

\[ \boldsymbol{ \left. +\frac{\partial}{\partial\chi}\left(asinh(\chi)sin(\theta_{2})\frac{\partial u}{\partial\chi}\right)\right] +\frac{1}{a^{2}sinh^{2}(\chi)sin^{2}(\theta_{2})}\frac{\partial{{}^2}u}{\partial\theta_{1}^{2}}} \]
\\
\\
\textbf{OBS: Não consegui simplificar $a^{3}sinh(\chi)sin(\theta_{2})\left[sinh^{2}(\chi)cos^{2}(\theta_{2})+cosh^{2}(\chi)sin^{2}(\theta_{2})\right]$.}

\section{Laplaciano hiperboloide de 2 folhas (4-dim)}

Dada a seguinte parametrização:

\begin{eqnarray*}
x_{4} & = & a\cosh\chi\cos\theta_{3}\\
x_{3} & = & a\sinh\chi\sin\theta_{3}\cos\theta_{2}\\
x_{2} & = & a\sinh\chi\sin\theta_{3}\sin\theta_{2}\cos\theta_{1}\\
x_{1} & = & a\sinh\chi\sin\theta_{3}\sin\theta_{2}\sin\theta_{1}\\
0 & \leq & \chi<\infty,\ 0\leq\theta_{3},\theta_{2}\leq\pi,\ 0\leq\theta_{1}<2\pi
\end{eqnarray*}

com

\[
q_{1}=\theta_{1},\ q_{2}=\theta_{2},\ q_{3}=\theta_{3},\,q_{4}=\chi
\]

Utiliza-se o Operador de Laplace-Beltrami para encontrar o Laplaciano:

\[
\Delta u=\frac{1}{\sqrt{\left\vert g\right\vert }}\frac{\partial}{\partial q_{i}}\left(\sqrt{\left\vert g\right\vert }g^{ij}\frac{\partial u}{\partial q_{j}}\right)
\]

De acordo com Arfken(2017), pode-se determinar o tensor métrico $g_{ij}$

\[
G=(g_{ij})=\underset{k}{\sum}\frac{\partial x_{k}}{\partial q_{_{^{i}}}}\frac{\partial x_{^{_{k}}}}{\partial q_{_{^{j}}}}
\]

Na dada parametrização, forma-se um \textbf{sistema de coordenadas
ortogonal} de modo que, segundo Riley(2006), os elementos do tensor
métrico podem ser dados por:
\[
g_{ij}=\begin{cases}
h_{i}^{2} & ,i=j\\
0 & ,i\neq j
\end{cases}
\]

Onde $h_{i}^{2}=\sum_{j}\left[\left(\frac{\partial x_{j}}{\partial q_{i}}\right)^{2}\right]$.
Ou seja, $h_{i}$ é associado a $q_{i}$. Como nesse caso $q_{1}=\theta_{1},\thinspace q_{2}=\theta_{2},\thinspace q_{3}=\theta_{3},\,q_{4}=\chi$
então tem-se $h_{1}=h_{\theta_{!}}$, $h_{2}=h_{\theta_{2}}$, $h_{3}=h_{\theta_{3}}$
e $h_{4}=\chi$.

Ou seja, a matriz $G=(g_{ij})$ é diagonal com os elementos $h_{1}^{2}$,
$h_{2}^{2}$ , $h_{3}^{2}$ e $h_{4}^{2}$. Em outras palavras $G=diag(h_{1}^{2},\,h_{2}^{2},\,h_{3}^{2},\,h_{4}^{2})$.

\[
G=(g_{ij})=\left(\begin{array}{cccc}
h_{1}^{2} & 0 & 0 & 0\\
0 & h_{2}^{2} & 0 & 0\\
0 & 0 & h_{3}^{2} & 0\\
0 & 0 & 0 & h_{4}^{2}
\end{array}\right)
\]

Ainda de acordo com Riley(2006), o tensor métrico inverso, para um
sistema ortogonal de coordenadas, pode ser dado por:

\[
G^{-1}=(g^{ij})=\begin{cases}
1/h_{i}^{2} & ,i=j\\
0 & ,i\neq j
\end{cases}
\]

\[
G^{-1}=(g^{ij})=\left(\begin{array}{cccc}
\frac{1}{h_{1}^{2}} & 0 & 0 & 0\\
0 & \frac{1}{h_{2}^{2}} & 0 & 0\\
0 & 0 & \frac{1}{h_{3}^{2}} & 0\\
0 & 0 & 0 & \frac{1}{h_{4}^{2}}
\end{array}\right)
\]

E, por fim, $g=det(g_{ij})=h_{1}^{2}h_{2}^{2}h_{3}^{2}h_{4}^{2}$.
Consequentemente, $\sqrt{|g|}=h_{1}h_{2}h_{3}h_{4}$.

Os coeficientes métricos são encontrados da seguinte forma:

\[
h_{1}^{2}=h_{\theta_{1}}^{2}=\left(\frac{\partial x_{1}}{\partial\theta_{1}}\right)^{2}+\left(\frac{\partial x_{2}}{\partial\theta_{1}}\right)^{2}+\left(\frac{\partial x_{3}}{\partial\theta_{1}}\right)^{2}+\left(\frac{\partial x_{4}}{\partial\theta_{1}}\right)^{2}
\]

\[
h_{2}^{2}=h_{\theta_{2}}^{2}=\left(\frac{\partial x_{1}}{\partial\theta_{2}}\right)^{2}+\left(\frac{\partial x_{2}}{\partial\theta_{2}}\right)^{2}+\left(\frac{\partial x_{3}}{\partial\theta_{2}}\right)^{2}+\left(\frac{\partial x_{4}}{\partial\theta_{2}}\right)^{2}
\]

\[
h_{3}^{2}=h_{\theta_{3}}^{2}=\left(\frac{\partial x_{1}}{\partial\theta_{3}}\right)^{2}+\left(\frac{\partial x_{2}}{\partial\theta_{3}}\right)^{2}+\left(\frac{\partial x_{3}}{\partial\theta_{3}}\right)^{2}+\left(\frac{\partial x_{4}}{\partial\theta_{3}}\right)^{2}
\]

\[
h_{4}^{2}=h_{\chi}^{2}=\left(\frac{\partial x_{1}}{\partial\chi}\right)^{2}+\left(\frac{\partial x_{2}}{\partial\chi}\right)^{2}+\left(\frac{\partial x_{3}}{\partial\chi}\right)^{2}+\left(\frac{\partial x_{4}}{\partial\chi}\right)^{2}
\]

Calcula-se tais coeficientes:

\[
h_{1}^{2}=h_{\theta_{1}}^{2}=\left(\frac{\partial x_{1}}{\partial\theta_{1}}\right)^{2}+\left(\frac{\partial x_{2}}{\partial\theta_{1}}\right)^{2}=a^{2}sinh^{2}(\chi)cos^{2}(\theta_{1})sin^{2}(\theta_{2})sin^{2}(\theta_{3})+a^{2}sinh^{2}(\chi)sin^{2}(\theta_{1})sin^{2}(\theta_{2})sin^{2}(\theta_{3})
\]

\[
=a^{2}sinh^{2}(\chi)sin^{2}(\theta_{2})sin^{2}(\theta_{3})\left[cos^{2}(\theta_{1})+sin^{2}(\theta_{1})\right]
\]

\[
=a^{2}sinh^{2}(\chi)sin^{2}(\theta_{2})sin^{2}(\theta_{3})
\]

\[
\boldsymbol{h_{1}^{2}=h_{\theta_{1}}^{2}=a^{2}sinh^{2}(\chi)sin^{2}(\theta_{2})sin^{2}(\theta_{3})}
\]
\\
\\
\[
h_{2}^{2}=h_{\theta_{2}}^{2}=\left(\frac{\partial x_{1}}{\partial\theta_{2}}\right)^{2}+\left(\frac{\partial x_{2}}{\partial\theta_{2}}\right)^{2}+\left(\frac{\partial x_{3}}{\partial\theta_{2}}\right)^{2}=a^{2}sinh^{2}(\chi)sin^{2}(\theta_{1})cos^{2}(\theta_{2})sin^{2}(\theta_{3})+a^{2}sinh^{2}(\chi)cos^{2}(\theta_{1})cos^{2}(\theta_{2})sin^{2}(\theta_{3})
\]

\[
+a^{2}sinh^{2}(\chi)sin^{2}(\theta_{2})sin^{2}(\theta_{3})
\]

\[
=a^{2}sinh^{2}(\chi)cos^{2}(\theta_{2})sin^{2}(\theta_{3})\left[sin^{2}(\theta_{1})+cos^{2}(\theta_{1})\right]+a^{2}sinh^{2}(\chi)sin^{2}(\theta_{2})sin^{2}(\theta_{3})
\]

\[
=a^{2}sinh^{2}(\chi)cos^{2}(\theta_{2})sin^{2}(\theta_{3})+a^{2}sinh^{2}(\chi)sin^{2}(\theta_{2})sin^{2}(\theta_{3})
\]

\[
=a^{2}sinh^{2}(\chi)sin^{2}(\theta_{3})\left[cos^{2}(\theta_{2})+sin^{2}(\theta_{2})\right]
\]

\[
\boldsymbol{h_{2}^{2}=h_{\theta_{2}}^{2}=a^{2}sinh^{2}(\chi)sin^{2}(\theta_{3})}
\]
\\
\\
\[
h_{3}^{2}=h_{\theta_{3}}^{2}=\left(\frac{\partial x_{1}}{\partial\theta_{3}}\right)^{2}+\left(\frac{\partial x_{2}}{\partial\theta_{3}}\right)^{2}+\left(\frac{\partial x_{3}}{\partial\theta_{3}}\right)^{2}+\left(\frac{\partial x_{4}}{\partial\theta_{3}}\right)^{2}=a^{2}sinh^{2}(\chi)sin^{2}(\theta_{1})sin^{2}(\theta_{2})cos^{2}(\theta_{3})
\]

\[ +a^{2}sinh^{2}(\chi)cos^{2}(\theta_{1})sin^{2}(\theta_{2})cos^{2}(\theta_{3})\]

\[
+a^{2}sinh^{2}(\chi)cos^{2}(\theta_{2})cos^{2}(\theta_{3})+a^{2}cosh^{2}(\chi)sin^{2}(\theta_{3})
\]

\[
=a^{2}sinh^{2}(\chi)sin^{2}(\theta_{2})cos^{2}(\theta_{3})\left[sin^{2}(\theta_{1})+cos^{2}(\theta_{1})\right]+a^{2}sinh^{2}(\chi)cos^{2}(\theta_{2})cos^{2}(\theta_{3})+a^{2}cosh^{2}(\chi)sin^{2}(\theta_{3})
\]

\[
=a^{2}sinh^{2}(\chi)sin^{2}(\theta_{2})cos^{2}(\theta_{3})+a^{2}sinh^{2}(\chi)cos^{2}(\theta_{2})cos^{2}(\theta_{3})+a^{2}cosh^{2}(\chi)sin^{2}(\theta_{3})
\]

\[
=a^{2}sinh^{2}(\chi)cos^{2}(\theta_{3})\left[sin^{2}(\theta_{2})+cos^{2}(\theta_{2})\right]+a^{2}cosh^{2}(\chi)sin^{2}(\theta_{3})
\]

\[
=a^{2}sinh^{2}(\chi)cos^{2}(\theta_{3})+a^{2}cosh^{2}(\chi)sin^{2}(\theta_{3})
\]

\[
\boldsymbol{h_{3}^{2}=h_{\theta_{3}}^{2}=a^{2}\left[sinh^{2}(\chi)cos^{2}(\theta_{3})+cosh^{2}(\chi)sin^{2}(\theta_{3})\right]}
\]
\\
\\
\[
h_{4}^{2}=h_{\chi}^{2}=\left(\frac{\partial x_{1}}{\partial\chi}\right)^{2}+\left(\frac{\partial x_{2}}{\partial\chi}\right)^{2}+\left(\frac{\partial x_{3}}{\partial\chi}\right)^{2}+\left(\frac{\partial x_{4}}{\partial\chi}\right)^{2}=a^{2}cosh^{2}(\chi)sin^{2}(\theta_{1})sin^{2}(\theta_{2})sin^{2}(\theta_{3})
\]

\[ +a^{2}cosh^{2}(\chi)cos^{2}(\theta_{1})sin^{2}(\theta_{2})sin^{2}(\theta_{3}) \]

\[
+a^{2}cosh^{2}(\chi)cos^{2}(\theta_{2})sin^{2}(\theta_{3})+a^{2}sinh^{2}(\chi)cos^{2}(\theta_{3})
\]

\[
=a^{2}cosh^{2}(\chi)sin^{2}(\theta_{2})sin^{2}(\theta_{3})\left[sin^{2}(\theta_{1})+cos^{2}(\theta_{1})\right]+a^{2}cosh^{2}(\chi)cos^{2}(\theta_{2})sin^{2}(\theta_{3})+a^{2}sinh^{2}(\chi)cos^{2}(\theta_{3})
\]

\[
=a^{2}cosh^{2}(\chi)sin^{2}(\theta_{2})sin^{2}(\theta_{3})+a^{2}cosh^{2}(\chi)cos^{2}(\theta_{2})sin^{2}(\theta_{3})+a^{2}sinh^{2}(\chi)cos^{2}(\theta_{3})
\]

\[
=a^{2}cosh^{2}(\chi)sin^{2}(\theta_{3})\left[sin^{2}(\theta_{2})+cos^{2}(\theta_{2})\right]+a^{2}sinh^{2}(\chi)cos^{2}(\theta_{3})
\]

\[
=a^{2}cosh^{2}(\chi)sin^{2}(\theta_{3})+a^{2}sinh^{2}(\chi)cos^{2}(\theta_{3})
\]

\begin{equation}
\boldsymbol{h_{4}^{2}=h_{\chi}^{2}=a^{2}\left[cosh^{2}(\chi)sin^{2}(\theta_{3})+sinh^{2}(\chi)cos^{2}(\theta_{3})\right]}
\end{equation}
\\
\\
Observa-se que $h_{3}^{2}=h_{\theta_{3}}^{2}=h_{4}^{2}=h_{\chi}^{2}$
então $\sqrt{|g|}=h_{1}h_{2}h_{3}^{2}$. Agora é possível construir
a matriz $\sqrt{|g|}g^{ij}$:

\[
\sqrt{|g|}g^{ij}=h_{1}h_{2}h_{3}^{2}\left(\begin{array}{cccc}
\frac{1}{h_{1}^{2}} & 0 & 0 & 0\\
0 & \frac{1}{h_{2}^{2}} & 0 & 0\\
0 & 0 & \frac{1}{h_{3}^{2}} & 0\\
0 & 0 & 0 & \frac{1}{h_{4}^{2}}
\end{array}\right)=\left(\begin{array}{cccc}
\frac{h_{2}h_{3}^{2}}{h_{1}} & 0 & 0 & 0\\
0 & \frac{h_{1}h_{3}^{2}}{h_{2}} & 0 & 0\\
0 & 0 & h_{1}h_{2} & 0\\
0 & 0 & 0 & h_{1}h_{2}
\end{array}\right)
\]

Aplica-se os resultados no Operador de Laplace-Beltrami e chega-se
em:

\[
\Delta u=\frac{1}{\sqrt{\left\vert g\right\vert }}\left[\frac{\partial}{\partial q_{1}}\left(\sqrt{\left\vert g\right\vert }g^{11}\frac{\partial u}{\partial q_{1}}\right)+\frac{\partial}{\partial q_{2}}\left(\sqrt{\left\vert g\right\vert }g^{22}\frac{\partial u}{\partial q_{2}}\right)+\frac{\partial}{\partial q_{3}}\left(\sqrt{\left\vert g\right\vert }g^{33}\frac{\partial u}{\partial q_{3}}\right)++\frac{\partial}{\partial q_{4}}\left(\sqrt{\left\vert g\right\vert }g^{44}\frac{\partial u}{\partial q_{4}}\right)\right]
\]

\begin{align*}
\Delta u & =\frac{1}{h_{1}h_{2}h_{3}^{2}}\left[\frac{\partial}{\partial\theta_{1}}\left(\frac{h_{2}h_{3}^{2}}{h_{1}}\frac{\partial u}{\partial\theta_{1}}\right)+\frac{\partial}{\partial\theta_{2}}\left(\frac{h_{1}h_{3}^{2}}{h_{2}}\frac{\partial u}{\partial\theta_{2}}\right)+\frac{\partial}{\partial\theta_{3}}\left(h_{1}h_{2}\frac{\partial u}{\partial\theta_{3}}\right)+\frac{\partial}{\partial\chi}\left(h_{1}h_{2}\frac{\partial u}{\partial\chi}\right)\right]
\end{align*}

Dado que

\[
\frac{1}{h_{1}h_{2}h_{3}^{2}}=\frac{1}{a^{4}sinh^{2}(\chi)sin^{2}(\theta_{3})sin(\theta_{2})\left[sinh^{2}(\chi)cos^{2}(\theta_{3})+cosh^{2}(\chi)sin^{2}(\theta_{3})\right]}
\]

\[
\frac{h_{2}h_{3}^{2}}{h_{1}}=\frac{a^{2}\left[sinh^{2}(\chi)cos^{2}(\theta_{3})+cosh^{2}(\chi)sin^{2}(\theta_{3})\right]}{sin(\theta_{2})}
\]

\[
\frac{h_{1}h_{3}^{2}}{h_{2}}=sin(\theta_{2})a^{2}\left[sinh^{2}(\chi)cos^{2}(\theta_{3})+cosh^{2}(\chi)sin^{2}(\theta_{3})\right]
\]

\[
h_{1}h_{2}=a^{2}sinh^{2}(\chi)sin^{2}(\theta_{3})sin(\theta_{2})
\]

Vê-se que $\frac{h_{2}h_{3}^{2}}{h_{1}}$ não depende de $\theta_{1}$,
logo:

\[
\frac{\partial}{\partial\theta_{1}}\left(\frac{h_{2}h_{3}^{2}}{h_{1}}\frac{\partial u}{\partial\theta_{1}}\right)=\frac{h_{2}h_{3}^{2}}{h_{1}}\frac{\partial^{2}u}{\partial\theta_{1}^{2}}=\frac{a^{2}\left[sinh^{2}(\chi)cos^{2}(\theta_{3})+cosh^{2}(\chi)sin^{2}(\theta_{3})\right]}{sin(\theta_{2})}\frac{\partial^{2}u}{\partial\theta_{1}^{2}}
\]

O Laplaciano fica então:

\[
\Delta u=\frac{1}{h_{1}h_{2}h_{3}^{2}}\left[\frac{h_{2}h_{3}^{2}}{h_{1}}\frac{\partial^{2}u}{\partial\theta_{1}^{2}}+\frac{\partial}{\partial\theta_{2}}\left(\frac{h_{1}h_{3}^{2}}{h_{2}}\frac{\partial u}{\partial\theta_{2}}\right)+\frac{\partial}{\partial\theta_{3}}\left(h_{1}h_{2}\frac{\partial u}{\partial\theta_{3}}\right)+\frac{\partial}{\partial\chi}\left(h_{1}h_{2}\frac{\partial u}{\partial\chi}\right)\right]
\]

\[
\Delta u=\frac{1}{h_{1}h_{2}h_{3}^{2}}\left[\frac{\partial}{\partial\theta_{2}}\left(\frac{h_{1}h_{3}^{2}}{h_{2}}\frac{\partial u}{\partial\theta_{2}}\right)+\frac{\partial}{\partial\theta_{3}}\left(h_{1}h_{2}\frac{\partial u}{\partial\theta_{3}}\right)+\frac{\partial}{\partial\chi}\left(h_{1}h_{2}\frac{\partial u}{\partial\chi}\right)\right]+\frac{1}{h_{1}^{2}}\frac{\partial^{2}u}{\partial\theta_{1}^{2}}
\]

Para finalmente encontrar a expressão do Laplaciano, aplica-se as
expressões para dos coeficientes métricos:

\[
\boldsymbol{\Delta u=\frac{1}{a^{4}sinh^{2}(\chi)sin^{2}(\theta_{3})sin(\theta_{2})\left[sinh^{2}(\chi)cos^{2}(\theta_{3})+cosh^{2}(\chi)sin^{2}(\theta_{3})\right]}}
\]
\[ \boldsymbol{\left[\frac{\partial}{\partial\theta_{2}}\left(sin(\theta_{2})a^{2}\left[sinh^{2}(\chi)cos^{2}(\theta_{3})+cosh^{2}(\chi)sin^{2}(\theta_{3})\right]\frac{\partial u}{\partial\theta_{2}}\right) \right.}\]
\[
\boldsymbol{\left. +\frac{\partial}{\partial\theta_{3}}\left(a^{2}sinh^{2}(\chi)sin^{2}(\theta_{3})sin(\theta_{2})\frac{\partial u}{\partial\theta_{3}}\right)+\frac{\partial}{\partial\chi}\left(a^{2}sinh^{2}(\chi)sin^{2}(\theta_{3})sin(\theta_{2})\frac{\partial u}{\partial\chi}\right)\right]}
\]

\[
\boldsymbol{+\frac{1}{a^{2}sinh^{2}(\chi)sin^{2}(\theta_{2})sin^{2}(\theta_{3})}\frac{\partial^{2}u}{\partial\theta_{1}^{2}}}
\]

\textbf{OBS1: Não consegui simplificar $a^{4}sinh^{2}(\chi)sin^{2}(\theta_{3})sin(\theta_{2})\left[sinh^{2}(\chi)cos^{2}(\theta_{3})+cosh^{2}(\chi)sin^{2}(\theta_{3})\right]$.}

\textbf{OBS2: Utilizando outro Software de Matemática Simbólica (Python
utilizando o pacote SymPy ) eu encontro a expressão (1) simplificada
na forma $h_{3}^{2}=h_{\theta_{3}}^{2}=h_{4}^{2}=h_{\chi}^{2}=a^{2}\left[sin^{2}(\theta_{3})+sinh^{2}(\chi)\right]$.
Porém, fazendo à mão, eu não consegui entender como ele chegou nisso
e, portanto, não usei essa simplificação.}

\section{Comparando Resultados}

Ao comparar o \textbf{Laplaciano Hiperboloide de 1 Folha} (3-dim):

\[
\Delta u=\frac{1}{a^{3}cosh(\chi)\,sin(\theta_{2})\left[sinh^{2}(\chi)sin^{2}(\theta_{2})+cosh^{2}(\chi)cos^{2}(\theta_{2})\right]}\left[\frac{\partial}{\partial\theta_{2}}\left(a\,cosh(\chi)\,sin(\theta_{2})\frac{\partial u}{\partial\theta_{2}}\right) \right.
\]
\[ \left. +\frac{\partial}{\partial\chi}\left(a\,cosh(\chi)\,sin(\theta_{2})\frac{\partial u}{\partial\chi}\right)\right]+\frac{1}{a^{2}\,cosh^{2}(\chi)\,sin^{2}(\theta_{2})}\frac{\partial^{2}u}{\partial\theta_{1}^{2}}\]


Com o \textbf{Laplaciano Hiperboloide de 2 Folhas} (3-dim):

\[
\Delta u=\frac{1}{a^{3}sinh(\chi)sin(\theta_{2})\left[sinh^{2}(\chi)cos^{2}(\theta_{2})+cosh^{2}(\chi)sin^{2}(\theta_{2})\right]}\left[\frac{\partial}{\partial\theta_{2}}\left(a\,sinh(\chi)\,sin(\theta_{2})\frac{\partial u}{\partial\theta_{2}}\right) \right.
\]
\[ \left. +\frac{\partial}{\partial\chi}\left(a\,sinh(\chi)\,sin(\theta_{2})\frac{\partial u}{\partial\chi}\right)\right] +\frac{1}{a^{2}\,sinh^{2}(\chi)\,sin^{2}(\theta_{2})}\frac{\partial{{}^2}u}{\partial\theta_{1}^{2}} \]


Percebe-se que a principal diferença são algumas trocas de $sin(\theta_{2})$
por $cos(\theta_{2})$ e, principalmente, de $cosh(\chi)$ por $sinh(\chi)$.

Quando compara-se o \textbf{Laplaciano Hiperboloide de 1 Folha} (4-dim) 

\[ \Delta u=\frac{1}{a^{4}sin(\theta_{2})sin^{2}(\theta_{3})cosh^{2}(\chi)[\cosh^{2}\left(\chi\right)-\sin^{2}\left(\theta_{3}\right)]}\left[\frac{\partial}{\partial\theta_{2}}\left(-a^{2}sin(\theta_{2})[\sin^{2}\left(\theta_{3}\right)-\cosh^{2}\left(\chi\right)]\frac{\partial u}{\partial\theta_{2}}\right) \right. \]
\[ \left. +\frac{\partial}{\partial\theta_{3}}\left(a^{2}\sin\left(\theta_{2}\right)\sin^{2}\left(\theta_{3}\right)\cosh^{2}\left(\chi\right)\frac{\partial u}{\partial\theta_{3}}\right) +\frac{\partial}{\partial\chi}\left(a^{2}\sin\left(\theta_{2}\right)\sin^{2}\left(\theta_{3}\right)\cosh^{2}\left(\chi\right)\frac{\partial u}{\partial\chi}\right) \right] \]

\[ -\left(\frac{1}{a^{2}sin^{2}(\theta_{2})sin^{3}(\theta_{3})cosh^{2}(\chi)}\right)\frac{\partial^{2}u}{\partial\theta_{1}^{2}} \]

Com o \textbf{Laplaciano Hiperboloide de 2 Folha} (4-dim).
\[
\Delta u=\frac{1}{a^{4}sinh^{2}(\chi)sin^{2}(\theta_{3})sin(\theta_{2})\left[sinh^{2}(\chi)cos^{2}(\theta_{3})+cosh^{2}(\chi)sin^{2}(\theta_{3})\right]}
\]

\[ \left[\frac{\partial}{\partial\theta_{2}}\left(sin(\theta_{2})a^{2}\left[sinh^{2}(\chi)cos^{2}(\theta_{3})+cosh^{2}(\chi)sin^{2}(\theta_{3})\right]\frac{\partial u}{\partial\theta_{2}}\right) \right.\]

\[
\left. +\frac{\partial}{\partial\theta_{3}}\left(a^{2}sinh^{2}(\chi)sin^{2}(\theta_{3})sin(\theta_{2})\frac{\partial u}{\partial\theta_{3}}\right)+\frac{\partial}{\partial\chi}\left(a^{2}sinh^{2}(\chi)sin^{2}(\theta_{3})sin(\theta_{2})\frac{\partial u}{\partial\chi}\right) \right]
\]

\[
+\frac{1}{a^{2}sinh^{2}(\chi)sin^{2}(\theta_{2})sin^{2}(\theta_{3})}\frac{\partial^{2}u}{\partial\theta_{1}^{2}}
\]

Percebe-se que a principal diferença são algumas trocas de $cosh(\chi)$
por $sinh(\chi)$.
\end{document}
