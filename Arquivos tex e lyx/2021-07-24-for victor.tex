
\documentclass[a4paper,12pt]{article}
%%%%%%%%%%%%%%%%%%%%%%%%%%%%%%%%%%%%%%%%%%%%%%%%%%%%%%%%%%%%%%%%%%%%%%%%%%%%%%%%%%%%%%%%%%%%%%%%%%%%%%%%%%%%%%%%%%%%%%%%%%%%%%%%%%%%%%%%%%%%%%%%%%%%%%%%%%%%%%%%%%%%%%%%%%%%%%%%%%%%%%%%%%%%%%%%%%%%%%%%%%%%%%%%%%%%%%%%%%%%%%%%%%%%%%%%%%%%%%%%%%%%%%%%%%%%
\usepackage{amssymb}
\usepackage{amsfonts}
\usepackage{graphicx}
\usepackage{amsmath}

\setcounter{MaxMatrixCols}{10}
%TCIDATA{OutputFilter=LATEX.DLL}
%TCIDATA{Version=5.50.0.2953}
%TCIDATA{<META NAME="SaveForMode" CONTENT="1">}
%TCIDATA{BibliographyScheme=Manual}
%TCIDATA{Created=Sat Oct 14 01:01:49 2006}
%TCIDATA{LastRevised=Saturday, July 24, 2021 16:51:15}
%TCIDATA{<META NAME="GraphicsSave" CONTENT="32">}
%TCIDATA{<META NAME="DocumentShell" CONTENT="Journal Articles\Standard LaTeX Article">}
%TCIDATA{CSTFile=LaTeX article (bright).cst}

\textheight=26.5cm
\textwidth=18cm
\oddsidemargin=-1.0cm
\evensidemargin=-1.0cm
\topmargin=-2.5cm
\def\baselinestretch{1.6}
\input{tcilatex}
\begin{document}


;

\section{de solu\c{c}\~{a}o}

;

vamos passar para solu\c{c}\~{a}o agora. come\c{c}amos de eq. de Laplace
para hiperb de 1 folha em 3 e 4 dim

- leva as suas express\~{o}es para Laplacianos na forma seguinte:

;

o Laplaciano nas coordenadas esferoidais achatadas em 3 dim

\begin{eqnarray*}
\nabla ^{2}u &=&\frac{1}{a^{2}\left( \sinh ^{2}\chi +\cos ^{2}\theta
_{2}\right) }\left[ \frac{1}{\sin \theta _{2}}\frac{\partial }{\partial
\theta _{2}}\left( \sin \theta _{2}\frac{\partial u}{\partial \theta _{2}}%
\right) +\frac{1}{\cosh \chi }\frac{\partial }{\partial \chi }\left( \cosh
\chi \frac{\partial u}{\partial \chi }\right) \right]  \\
&&+\frac{1}{a^{2}\sin ^{2}\theta _{2}\cosh ^{2}\chi }\frac{\partial ^{2}u}{%
\partial \theta _{1}^{2}}\ \left( 3\ \dim \right) 
\end{eqnarray*}

o Laplaciano nas coordenadas esferoidais achatadas em 4 dim%
\begin{eqnarray*}
\nabla ^{2}u &=&\frac{1}{a^{2}\left( \sinh ^{2}\chi +\cos ^{2}\theta
_{3}\right) }\left( \frac{1}{\sin ^{2}\theta _{3}}\frac{\partial }{\partial
\theta _{3}}\left( \sin ^{2}\theta _{3}\frac{\partial u}{\partial \theta _{3}%
}\right) +\frac{1}{\cosh ^{2}\chi }\frac{\partial }{\partial \chi }\left(
\cosh ^{2}\chi \frac{\partial u}{\partial \chi }\right) \right)  \\
&&+\frac{1}{a^{2}\sin ^{2}\theta _{3}\cosh ^{2}\chi }\nabla _{S^{2}}^{2}u\
\left( 4\ \dim \right) 
\end{eqnarray*}%
onde 
\begin{equation*}
\nabla _{S^{2}}^{2}u=\left[ \left( \frac{1}{\sin \theta _{2}}\frac{\partial 
}{\partial \theta _{2}}\left( \sin \theta _{2}\frac{\partial u}{\partial
\theta _{2}}\right) \right) +\frac{1}{\sin ^{2}\theta _{3}}\frac{\partial
^{2}u}{\partial \theta _{1}^{2}}\right] 
\end{equation*}%
\'{e} o Laplaciano na esfera $S^{2}$

;

separa\c{c}\~{a}o de vari\'{a}veis, ainda para 4d

aplicar o troco de vari\'{a}veis

\begin{equation*}
\xi =\cos \theta _{3},\ \zeta =\sinh \chi ,\ \theta _{1},\theta _{2}
\end{equation*}%
\begin{equation*}
-1\leq \xi \leq 1,\ 0\leq \zeta <\infty ,\ 0\leq \theta _{1}<2\pi ,\ 0\leq
\theta _{2},\theta _{3}<\pi 
\end{equation*}%
e separar a parte com $\left( \theta _{1},\theta _{2}\right) $ e a parte com 
$\left( \theta _{3},\chi \right) $

;

\section{caso de 3 dim}

;%
\begin{eqnarray*}
x_{3} &=&a\sinh \chi \cos \theta _{2} \\
x_{2} &=&a\cosh \chi \sin \theta _{2}\cos \theta _{1} \\
x_{1} &=&a\cosh \chi \sin \theta _{2}\sin \theta _{1} \\
0 &\leq &\chi <\infty ,\ 0\leq \theta _{2}\leq \pi ,\ 0\leq \theta _{1}<2\pi 
\end{eqnarray*}

\begin{eqnarray*}
\nabla ^{2}u &=&\frac{1}{a^{2}\left( \sinh ^{2}\chi +\cos ^{2}\theta
_{2}\right) }\left[ \frac{1}{\sin \theta _{2}}\frac{\partial }{\partial
\theta _{2}}\left( \sin \theta _{2}\frac{\partial u}{\partial \theta _{2}}%
\right) +\frac{1}{\cosh \chi }\frac{\partial }{\partial \chi }\left( \cosh
\chi \frac{\partial u}{\partial \chi }\right) \right]  \\
&&+\frac{1}{a^{2}\sin ^{2}\theta _{2}\cosh ^{2}\chi }\frac{\partial ^{2}u}{%
\partial \theta _{1}^{2}}=0
\end{eqnarray*}%
aplica separa\c{c}\~{a}o de vari\'{a}veis na eq.

1ro separar a variavel $\theta _{1}$ com a constante de separa\c{c}\~{a}o $%
\lambda _{1}$

depois separar as vari\'{a}veis $\theta _{2}$ e $\chi $ a constante de separa%
\c{c}\~{a}o $\lambda _{2}$

deve obter 3 equa\c{c}\~{o}es dif. ordinarias com constantes de separa\c{c}%
\~{a}o $\lambda _{1}$ e $\lambda _{2}$

aplicar as condi\c{c}\~{o}es de contorno: 
\begin{equation*}
u\left( \theta _{1},\theta _{2},\chi \right) =u\left( \theta _{1}+2\pi
,\theta _{2},\chi \right) 
\end{equation*}%
periodicidade (que segue da univocacidade) pela vari\'{a}vel $\theta _{1}$

e a condi\c{c}\~{a}o que solu\c{c}\~{a}o deve ser n\~{a}o singular dentro de
intervalos de varia\c{c}\~{a}o das vari\'{a}veis

;

\section{caso de 4 dim}

;

ser\'{a} o proximo passo

\end{document}
