\documentclass{article}
\usepackage[utf8]{inputenc}
\usepackage[round]{natbib}
\usepackage{hyperref}
\usepackage[brazil]{babel}
\usepackage{indentfirst}
\usepackage{graphicx}
\usepackage{amsmath}
\usepackage{amssymb}
\usepackage[left=1cm,top=1cm,right=1cm,bottom=1cm]{geometry}
\usepackage{esint}

\begin{document}
\[
\nabla^{2}u=\frac{1}{a^{2}\left(\sinh^{2}\chi+\cos^{2}\theta_{3}\right)}\left(\frac{1}{\sin^{2}\theta_{3}}\frac{\partial}{\partial\theta_{3}}\left(\sin^{2}\theta_{3}\frac{\partial u}{\partial\theta_{3}}\right)+\frac{1}{\cosh^{2}\chi}\frac{\partial}{\partial\chi}\left(\cosh^{2}\chi\frac{\partial u}{\partial\chi}\right)\right)+\frac{1}{a^{2}\sin^{2}\theta_{3}\cosh^{2}\chi}\nabla_{S^{2}}^{2}u=0
\]

Tal que

\[
\nabla_{S^{2}}^{2}u=\left[\left(\frac{1}{\sin\theta_{2}}\frac{\partial}{\partial\theta_{2}}\left(\sin\theta_{2}\frac{\partial u}{\partial\theta_{2}}\right)\right)+\frac{1}{\sin^{2}\theta_{2}}\frac{\partial^{2}u}{\partial\theta_{1}^{2}}\right]
\]

� o Laplaciano na esfera $S^{2}$.

Para realizar a separa��o de vari�veis fa�amos

\[
u(\chi,\theta_{1},\theta_{2},\theta_{3})=Y(\theta_{1},\theta_{2})W(\chi,\theta_{3})
\]

Aplicando na Equa��o\\

\[
\nabla^{2}u=\frac{Y}{a^{2}\left(\sinh^{2}\chi+\cos^{2}\theta_{3}\right)}\left(\frac{1}{\sin^{2}\theta_{3}}\frac{\partial}{\partial\theta_{3}}\left(\sin^{2}\theta_{3}\frac{\partial W}{\partial\theta_{3}}\right)+\frac{1}{\cosh^{2}\chi}\frac{\partial}{\partial\chi}\left(\cosh^{2}\chi\frac{\partial W}{\partial\chi}\right)\right)+\frac{W}{a^{2}\sin^{2}\theta_{3}\cosh^{2}\chi}\left(\nabla_{S^{2}}^{2}Y\right)=0
\]
\\
Multiplicando os dois lados por $\frac{a^{2}\sin^{2}\theta_{3}\cosh^{2}\chi}{WY}$\\
\[
\frac{\sin^{2}\theta_{3}\cosh^{2}\chi}{W\left(\sinh^{2}\chi+\cos^{2}\theta_{3}\right)}\left(\frac{1}{\sin^{2}\theta_{3}}\frac{\partial}{\partial\theta_{3}}\left(\sin^{2}\theta_{3}\frac{\partial W}{\partial\theta_{3}}\right)+\frac{1}{\cosh^{2}\chi}\frac{\partial}{\partial\chi}\left(\cosh^{2}\chi\frac{\partial W}{\partial\chi}\right)\right)+\frac{1}{Y}\left(\nabla_{S^{2}}^{2}Y\right)=0
\]
\\
Ent�o, tem-se\\
\[
\frac{\sin^{2}\theta_{3}\cosh^{2}\chi}{W\left(\sinh^{2}\chi+\cos^{2}\theta_{3}\right)}\left(\frac{1}{\sin^{2}\theta_{3}}\frac{\partial}{\partial\theta_{3}}\left(\sin^{2}\theta_{3}\frac{\partial W}{\partial\theta_{3}}\right)+\frac{1}{\cosh^{2}\chi}\frac{\partial}{\partial\chi}\left(\cosh^{2}\chi\frac{\partial W}{\partial\chi}\right)\right)=-\frac{1}{Y}\left(\nabla_{S^{2}}^{2}Y\right)=\lambda_{1}
\]
\\
De modo que chega-se em duas EDPs\\
\[
\frac{\sin^{2}\theta_{3}\cosh^{2}\chi}{W\left(\sinh^{2}\chi+\cos^{2}\theta_{3}\right)}\left(\frac{1}{\sin^{2}\theta_{3}}\frac{\partial}{\partial\theta_{3}}\left(\sin^{2}\theta_{3}\frac{\partial W}{\partial\theta_{3}}\right)+\frac{1}{\cosh^{2}\chi}\frac{\partial}{\partial\chi}\left(\cosh^{2}\chi\frac{\partial W}{\partial\chi}\right)\right)=\lambda_{1}
\]

\[
\nabla_{S^{2}}^{2}Y=-Y\lambda_{1}
\]
Assumindo 

\[
Y(\theta_{1},\theta_{2})=\Theta_{1}(\theta_{1})\Theta_{2}(\theta_{2})
\]

\[
W(\chi,\theta_{3})=X(\chi)\Theta_{3}(\theta_{3})
\]
\\
Tem-se

\section{Parte Esf�rica}

A equa��o $\nabla_{S^{2}}^{2}Y=-Y\lambda_{1}$ fica da seguinte forma\\
\[
\nabla_{S^{2}}^{2}Y=\left[\left(\frac{1}{\sin\theta_{2}}\frac{\partial}{\partial\theta_{2}}\left(\sin\theta_{2}\frac{\partial Y}{\partial\theta_{2}}\right)\right)+\frac{1}{\sin^{2}\theta_{2}}\frac{\partial^{2}Y}{\partial\theta_{1}^{2}}\right]=-Y\lambda_{1}
\]
\\
A solu��o dessa equa��o � expressa em termos de harm�nicos esf�ricos.

\[
Y(\theta_{1},\theta_{2})=\Theta_{1}(\theta_{1})\Theta_{2}(\theta_{2})=Y_{n,\,m}(\theta_{1},\,\theta_{2})
\]
\\
Para determinarmos $\lambda_{1}$vamos desenvolver a express�o

\[
\frac{1}{\sin\theta_{2}}\frac{\partial}{\partial\theta_{2}}\left(\sin\theta_{2}\frac{\partial Y}{\partial\theta_{2}}\right)+\frac{1}{\sin^{2}\theta_{2}}\frac{\partial^{2}Y}{\partial\theta_{1}^{2}}+Y\lambda_{1}=0
\]

\[
Y=\Theta_{1}\Theta_{2}
\]
\\
Ent�o tem-se

\[
\frac{\Theta_{1}}{\sin\theta_{2}}\frac{d}{d\theta_{2}}\left(\sin\theta_{2}\frac{d\Theta_{2}}{d\theta_{2}}\right)+\frac{\Theta_{2}}{\sin^{2}\theta_{2}}\frac{d^{2}\Theta_{1}}{d\theta_{1}^{2}}+Y\lambda_{1}=0
\]
\\
Multiplicando por $\frac{\sin^{2}\theta_{2}}{\Theta_{1}\Theta_{2}}$

\[
\frac{\sin\theta_{2}}{\Theta_{2}}\frac{d}{d\theta_{2}}\left(\sin\theta_{2}\frac{d\Theta_{2}}{d\theta_{2}}\right)+\sin^{2}\theta_{2}\lambda_{1}+\frac{1}{\Theta_{1}}\frac{d^{2}\Theta_{1}}{d\theta_{1}^{2}}=0
\]
\\
Ent�o podemos separar

\[
\frac{\sin\theta_{2}}{\Theta_{2}}\frac{d}{d\theta_{2}}\left(\sin\theta_{2}\frac{d\Theta_{2}}{d\theta_{2}}\right)+\sin^{2}\theta_{2}\lambda_{1}=-\frac{1}{\Theta_{1}}\frac{d^{2}\Theta_{1}}{d\theta_{1}^{2}}=\lambda_{2}
\]
\\
Desse modo chega-se em 

\[
\frac{d^{2}\Theta_{1}}{d\theta_{1}^{2}}+\Theta_{1}\lambda_{2}=0\Rightarrow\Theta_{1,m}(\theta_{1})=Ae^{im\theta_{1}}+Be^{-im\theta_{1}},\boldsymbol{\,\lambda_{2}=m^{2}}
\]

\[
\frac{1}{\sin\theta_{2}}\frac{d}{d\theta_{2}}\left(\sin\theta_{2}\frac{d\Theta_{2}}{d\theta_{2}}\right)+\left(\lambda_{1}-\frac{m^{2}}{\sin^{2}\theta_{2}}\right)\Theta_{2}=0
\]
\\
A �ltima express�o � conhecida e dela obtemos $\lambda_{1}=n(n+1)$.

\section{Analisando a outra EDP}

\[
\frac{\sin^{2}\theta_{3}\cosh^{2}\chi}{W\left(\sinh^{2}\chi+\cos^{2}\theta_{3}\right)}\left(\frac{X}{\sin^{2}\theta_{3}}\frac{\partial}{\partial\theta_{3}}\left(\sin^{2}\theta_{3}\frac{\partial\Theta_{3}}{\partial\theta_{3}}\right)+\frac{\Theta_{3}}{\cosh^{2}\chi}\frac{\partial}{\partial\chi}\left(\cosh^{2}\chi\frac{\partial X}{\partial\chi}\right)\right)=\lambda_{1}
\]
\\
Multiplica-se os dois lados por $\frac{W\left(\sinh^{2}\chi+\cos^{2}\theta_{3}\right)}{X\Theta_{3}\sin^{2}\theta_{3}\cosh^{2}\chi}$\\
\[
\frac{1}{X\Theta_{3}}\left(\frac{X}{\sin^{2}\theta_{3}}\frac{\partial}{\partial\theta_{3}}\left(\sin^{2}\theta_{3}\frac{\partial\Theta_{3}}{\partial\theta_{3}}\right)+\frac{\Theta_{3}}{\cosh^{2}\chi}\frac{\partial}{\partial\chi}\left(\cosh^{2}\chi\frac{\partial X}{\partial\chi}\right)\right)=\lambda_{1}\frac{\left(\sinh^{2}\chi+\cos^{2}\theta_{3}\right)}{\sin^{2}\theta_{3}\cosh^{2}\chi}
\]

\[
\frac{1}{\Theta_{3}}\frac{1}{\sin^{2}\theta_{3}}\frac{\partial}{\partial\theta_{3}}\left(\sin^{2}\theta_{3}\frac{\partial\Theta_{3}}{\partial\theta_{3}}\right)+\frac{1}{X}\frac{1}{\cosh^{2}\chi}\frac{\partial}{\partial\chi}\left(\cosh^{2}\chi\frac{\partial X}{\partial\chi}\right)=\lambda_{1}\frac{\left(\sinh^{2}\chi+\cos^{2}\theta_{3}\right)}{\sin^{2}\theta_{3}\cosh^{2}\chi}
\]
Veja que\\
\[
\frac{\left(\sinh^{2}\chi+\cos^{2}\theta_{3}\right)}{\sin^{2}\theta_{3}\cosh^{2}\chi}=\frac{\sinh^{2}\chi}{\sin^{2}\theta_{3}\cosh^{2}\chi}+\frac{\cos^{2}\theta_{3}}{\sin^{2}\theta_{3}\cosh^{2}\chi}=\frac{\cosh^{2}\chi-1}{\sin^{2}\theta_{3}\cosh^{2}\chi}+\frac{1-\sin^{2}\theta_{3}}{\sin^{2}\theta_{3}\cosh^{2}\chi}
\]
\\
\[
=\frac{1}{\sin^{2}\theta_{3}}-\frac{1}{\sin^{2}\theta_{3}\cosh^{2}\chi}+\frac{1}{\sin^{2}\theta_{3}\cosh^{2}\chi}-\frac{1}{\cosh^{2}\chi}=\frac{1}{\sin^{2}\theta_{3}}-\frac{1}{\cosh^{2}\chi}
\]
\\
Ent�o a equa��o fica\\
\[
\frac{1}{\Theta_{3}}\frac{1}{\sin^{2}\theta_{3}}\frac{\partial}{\partial\theta_{3}}\left(\sin^{2}\theta_{3}\frac{\partial\Theta_{3}}{\partial\theta_{3}}\right)+\frac{1}{X}\frac{1}{\cosh^{2}\chi}\frac{\partial}{\partial\chi}\left(\cosh^{2}\chi\frac{\partial X}{\partial\chi}\right)=\lambda_{1}\frac{1}{\sin^{2}\theta_{3}}-\lambda_{1}\frac{1}{\cosh^{2}\chi}
\]
\\
\[
\frac{1}{\Theta_{3}}\frac{1}{\sin^{2}\theta_{3}}\frac{\partial}{\partial\theta_{3}}\left(\sin^{2}\theta_{3}\frac{\partial\Theta_{3}}{\partial\theta_{3}}\right)-\lambda_{1}\frac{1}{\sin^{2}\theta_{3}}+\frac{1}{X}\frac{1}{\cosh^{2}\chi}\frac{\partial}{\partial\chi}\left(\cosh^{2}\chi\frac{\partial X}{\partial\chi}\right)+\lambda_{1}\frac{1}{\cosh^{2}\chi}=0
\]

\[
\frac{1}{\Theta_{3}}\frac{1}{\sin^{2}\theta_{3}}\frac{\partial}{\partial\theta_{3}}\left(\sin^{2}\theta_{3}\frac{\partial\Theta_{3}}{\partial\theta_{3}}\right)-\lambda_{1}\frac{1}{\sin^{2}\theta_{3}}=-\frac{1}{X}\frac{1}{\cosh^{2}\chi}\frac{\partial}{\partial\chi}\left(\cosh^{2}\chi\frac{\partial X}{\partial\chi}\right)-\lambda_{1}\frac{1}{\cosh^{2}\chi}=\lambda_{2}
\]
Fica-se com\\
\[
-\frac{1}{\Theta_{3}}\frac{1}{\sin^{2}\theta_{3}}\frac{d}{d\theta_{3}}\left(\sin^{2}\theta_{3}\frac{d\Theta_{3}}{d\theta_{3}}\right)+\lambda_{1}\frac{1}{\sin^{2}\theta_{3}}=-\lambda_{3}
\]
\[
-\frac{1}{X}\frac{1}{\cosh^{2}\chi}\frac{d}{d\chi}\left(\cosh^{2}\chi\frac{dX}{d\chi}\right)-\lambda_{1}\frac{1}{\cosh^{2}\chi}=\lambda_{3}
\]
\\
Ou, de outra forma, tem-se\\

\[
-\frac{1}{\Theta_{3}}\frac{1}{\sin^{2}\theta_{3}}\frac{d}{d\theta_{3}}\left(\sin^{2}\theta_{3}\frac{d\Theta_{3}}{d\theta_{3}}\right)+\left(\frac{\lambda_{1}}{\sin^{2}\theta_{3}}+\lambda_{3}\right)=0
\]

\[
-\frac{1}{X}\frac{1}{\cosh^{2}\chi}\frac{d}{d\chi}\left(\cosh^{2}\chi\frac{dX}{d\chi}\right)-\left(\lambda_{3}+\frac{\lambda_{1}}{\cosh^{2}\chi}\right)=0
\]


\subsection{Equa��o para $\theta_{3}$}

\[
\frac{1}{\Theta_{3}}\frac{1}{\sin^{2}\theta_{3}}\frac{d}{d\theta_{3}}\left(\sin^{2}\theta_{3}\frac{d\Theta_{3}}{d\theta_{3}}\right)-\left(\lambda_{1}\frac{1}{\sin^{2}\theta_{3}}+\lambda_{3}\right)=0
\]

Aplicando troca de vari�veis $\xi=\cos\theta_{3}$ 
\[
sin^{2}(\theta_{3})=1-cos^{2}(\theta_{3})=1-\xi{}^{2}
\]

\[
\sin^{2}\theta_{3}\frac{d\Theta_{3}}{d\theta_{3}}=\sin^{2}\theta_{3}\frac{d\Theta_{3}}{d\xi}\frac{d\xi}{d\theta_{3}}=-\sin^{3}(\theta_{3})\frac{d\Theta_{3}}{d\xi}
\]

\[
\frac{d}{d\theta_{3}}\left(\sin^{2}\theta_{3}\frac{d\Theta_{3}}{d\theta_{3}}\right)=-\frac{d}{d\theta_{3}}\left(\sin^{3}(\theta_{3})\frac{d\Theta_{3}}{d\xi}\right)=-\left(3\cos(\theta_{3})\sin^{2}(\theta_{3})\frac{d\Theta_{3}}{d\xi}+\sin^{3}(\theta_{3})\frac{d}{d\theta_{3}}\frac{d\Theta_{3}}{d\xi}\right)
\]

\[
=-\left(3\cos(\theta_{3})\sin^{2}(\theta_{3})\frac{d\Theta_{3}}{d\xi}+\sin^{3}(\theta_{3})\frac{d}{d\xi}\frac{d\Theta_{3}}{d\theta_{3}}\right)=-\left(3\cos(\theta_{3})\sin^{2}(\theta_{3})\frac{d\Theta_{3}}{d\xi}-\sin^{4}(\theta_{3})\frac{d^{2}\Theta_{3}}{d\xi^{2}}\right)
\]

\[
=-\sin^{2}(\theta_{3})\left(3\cos(\theta_{3})\frac{d\Theta_{3}}{d\xi}-\sin^{2}(\theta_{3})\frac{d^{2}\Theta_{3}}{d\xi^{2}}\right)
\]

A equa��o fica\\
\[
-\frac{1}{\Theta_{3}}\left(3\cos(\theta_{3})\frac{d\Theta_{3}}{d\xi}-\sin^{2}(\theta_{3})\frac{d^{2}\Theta_{3}}{d\xi^{2}}\right)-\left(\lambda_{1}\frac{1}{\sin^{2}\theta_{3}}+\lambda_{3}\right)=0
\]

\[
\frac{1}{\Theta_{3}}\left(3\xi\frac{d\Theta_{3}}{d\xi}-\left(1-\xi{}^{2}\right)\frac{d^{2}\Theta_{3}}{d\xi^{2}}\right)+\left(\frac{n(n+1)}{\left(1-\xi{}^{2}\right)}+\lambda_{3}\right)=0
\]


\subsection{Equa��o para $\chi$}

\[
\frac{1}{X}\frac{1}{\cosh^{2}\chi}\frac{d}{d\chi}\left(\cosh^{2}\chi\frac{dX}{d\chi}\right)+\left(\lambda_{3}+\frac{\lambda_{1}}{\cosh^{2}\chi}\right)=0
\]

Aplicando $\zeta=\sinh\chi$

\[
\cosh^{2}\chi=1+\sinh\chi=1+\zeta^{2}
\]

\[
\cosh^{2}\chi\frac{dX}{d\chi}=\cosh^{2}\chi\frac{dX}{d\zeta}\frac{d\zeta}{d\chi}=\cosh^{3}\chi\frac{dX}{d\zeta}
\]

\[
\frac{d}{d\chi}\left(\cosh^{2}\chi\frac{dX}{d\chi}\right)=\frac{d}{d\chi}\left(\cosh^{3}\chi\frac{dX}{d\zeta}\right)=3\cosh^{2}\chi\sinh\chi\frac{dX}{d\zeta}+\cosh^{3}\chi\frac{d}{d\chi}\frac{dX}{d\zeta}
\]

\[
=3\cosh^{2}\chi\sinh\chi\frac{dX}{d\zeta}+\cosh^{3}\chi\frac{d}{d\zeta}\frac{dX}{d\chi}=3\cosh^{2}\chi\sinh\chi\frac{dX}{d\zeta}+\cosh^{4}\chi\frac{d^{2}X}{d\zeta^{2}}
\]

\[
=\cosh^{2}\chi\left(3\sinh\chi\frac{dX}{d\zeta}+\cosh^{2}\chi\frac{d^{2}X}{d\zeta^{2}}\right)
\]

A equa��o fica

\[
\frac{1}{X}\left(3\zeta\frac{dX}{d\zeta}+\left(1+\zeta^{2}\right)\frac{d^{2}X}{d\zeta^{2}}\right)+\left(\lambda_{3}+\frac{\lambda_{1}}{\left(1+\zeta^{2}\right)}\right)=0
\]

Agora aplicando $\alpha=i\zeta$

\[
-\alpha^{2}=\zeta^{2}
\]

\[
\frac{dX}{d\zeta}=\frac{dX}{d\alpha}\frac{d\alpha}{d\zeta}=i\frac{dX}{d\alpha}\Rightarrow3\zeta\frac{dX}{d\zeta}=3\zeta i\frac{dX}{d\alpha}=3\alpha\frac{dX}{d\alpha}
\]

\[
\frac{d^{2}X}{d\zeta^{2}}=\frac{d}{d\zeta}\frac{dX}{d\zeta}=i\frac{d}{d\zeta}\frac{dX}{d\alpha}=i\frac{d}{d\alpha}\frac{dX}{d\zeta}=-\frac{d}{d\alpha}\frac{dX}{d\alpha}=-\frac{d^{2}X}{d\alpha^{2}}
\]

Ent�o, tem-se

\[
\frac{1}{X}\left(3\alpha\frac{dX}{d\alpha}-\left(1-\alpha^{2}\right)\frac{d^{2}X}{d\alpha^{2}}\right)+\left(\lambda_{3}+\frac{n(n+1)}{\left(1-\alpha^{2}\right)}\right)=0
\]

\end{document}
