%% LyX 2.3.6.1 created this file.  For more info, see http://www.lyx.org/.
%% Do not edit unless you really know what you are doing.
\documentclass[english]{article}
\usepackage[T1]{fontenc}
\usepackage[latin9]{inputenc}
\usepackage{amsmath}
\usepackage{babel}
\begin{document}
Obter Laplaciano nas seguintes coordenadas:

\begin{eqnarray*}
x_{3} & = & a\sinh\chi\cos\theta_{2}\\
x_{2} & = & a\cosh\chi\sin\theta_{2}\cos\theta_{1}\\
x_{1} & = & a\cosh\chi\sin\theta_{2}\sin\theta_{1}\\
0 & \leq & \chi<\infty,\ 0\leq\theta_{2}\leq\pi,\ 0\leq\theta_{1}<2\pi
\end{eqnarray*}

E
\[
q_{1}=\theta_{1},\ q_{2}=\theta_{2},\ q_{3}=\chi
\]

Utilizando o Operador de Laplace-Beltrami

\[
\Delta u=\frac{1}{\sqrt{\left\vert g\right\vert }}\frac{\partial}{\partial q_{i}}\left(\sqrt{\left\vert g\right\vert }g^{ij}\frac{\partial u}{\partial q_{j}}\right)
\]

De acordo com Arfken(2017), podemos determinar o tensor m�trico $g_{ij}$

\[
G=(g_{ij})=\underset{k}{\sum}\frac{\partial x_{k}}{\partial q_{_{^{i}}}}\frac{\partial x_{^{_{k}}}}{\partial q_{_{^{j}}}}
\]

Por�m, de acordo com Riley(2006), se o sistema de coordenadas for
ortogonal tem-se:

\[
g_{ij}=\begin{cases}
h_{i}^{2} & ,i=j\\
0 & ,i\neq j
\end{cases}
\]

\[
G=(g_{ij})=\left(\begin{array}{ccc}
h_{1}^{2} & 0 & 0\\
0 & h_{2}^{2} & 0\\
0 & 0 & h_{3}^{2}
\end{array}\right)
\]

\[
g^{ij}=\begin{cases}
1/h_{i}^{2} & ,i=j\\
0 & ,i\neq j
\end{cases}
\]

E temos o determinante $g=det(g_{ij})=h_{1}^{2}h_{2}^{2}h_{3}^{2}$

Calculando os coeficientes m�tricos

\[
h_{1}^{2}=h_{\theta_{1}}^{2}=\left(\frac{\partial x_{1}}{\partial\theta_{1}}\right)^{2}+\left(\frac{\partial x_{2}}{\partial\theta_{1}}\right)^{2}+\left(\frac{\partial x_{3}}{\partial\theta_{1}}\right)^{2}=a^{2}cosh^{2}(\chi)sen^{2}(\theta_{2})
\]

\[
h_{2}^{2}=h_{\theta_{2}}^{2}=\left(\frac{\partial x_{1}}{\partial\theta_{2}}\right)^{2}+\left(\frac{\partial x_{2}}{\partial\theta_{2}}\right)^{2}+\left(\frac{\partial x_{3}}{\partial\theta_{2}}\right)^{2}=a^{2}senh^{2}(\chi)sen^{2}(\theta_{2})+a^{2}cosh^{2}(\chi)cos^{2}(\theta_{2})
\]

\[
h_{1}^{2}=h_{\chi}^{2}=\left(\frac{\partial x_{1}}{\partial\chi}\right)^{2}+\left(\frac{\partial x_{2}}{\partial\chi}\right)^{2}+\left(\frac{\partial x_{3}}{\partial\chi}\right)^{2}=a^{2}senh^{2}(\chi)sen^{2}(\theta_{2})+a^{2}cosh^{2}(\chi)cos^{2}(\theta_{2})
\]

Vemos que $h_{1}^{2}=h_{\chi}^{2}=h_{2}^{2}=h_{\theta_{2}}^{2}$.

Desse modo definimos

\[
G=\left(\begin{array}{ccc}
a^{2}cosh^{2}(\chi)sen^{2}(\theta_{2}) & 0 & 0\\
0 & a^{2}[senh^{2}(\chi)sen^{2}(\theta_{2})+cosh^{2}(\chi)cos^{2}(\theta_{2})] & 0\\
0 & 0 & a^{2}[senh^{2}(\chi)sen^{2}(\theta_{2})+cosh^{2}(\chi)cos^{2}(\theta_{2})]
\end{array}\right)
\]

\[
G^{-1}=\left(\begin{array}{ccc}
\frac{1}{a^{2}cosh^{2}(\chi)sen^{2}(\theta_{2})} & 0 & 0\\
0 & \frac{1}{a^{2}[senh^{2}(\chi)sen^{2}(\theta_{2})+cosh^{2}(\chi)cos^{2}(\theta_{2})]} & 0\\
0 & 0 & \frac{1}{a^{2}[senh^{2}(\chi)sen^{2}(\theta_{2})+cosh^{2}(\chi)cos^{2}(\theta_{2})]}
\end{array}\right)
\]

\[
g=det(G)=a^{2}cosh^{2}(\chi)sen^{2}(\theta_{2})\{a^{2}[senh^{2}(\chi)sen^{2}(\theta_{2})+cosh^{2}(\chi)cos^{2}(\theta_{2})]\}^{2}
\]

Agora, olhando para o operador de Laplace-Beltrami, precisamos determinar
$\sqrt{|g|}g^{ij}$

\[
\sqrt{|g|}=h_{1}h_{2}^{2}
\]

\[
\sqrt{|g|}g^{ij}=\left(\begin{array}{ccc}
\frac{h_{2}^{2}}{h_{1}} & 0 & 0\\
0 & h_{1} & 0\\
0 & 0 & h_{1}
\end{array}\right)
\]

\[
\sqrt{|g|}g^{ij}=\left(\begin{array}{ccc}
\frac{a\,[senh^{2}(\chi)\,sen^{2}(\theta_{2})+cosh^{2}(\chi)\,cos^{2}(\theta_{2})]}{cosh(\chi)\,sen(\theta_{2})} & 0 & 0\\
0 & a\,cosh(\chi)\,sen(\theta_{2}) & 0\\
0 & 0 & a\,cosh(\chi)\,sen(\theta_{2})
\end{array}\right)
\]

Ent�o, utilizando o operador de Laplace-Beltrami, chegamos na express�o
geral do Laplaciano
\[
\Delta u=\frac{1}{\sqrt{\left\vert g\right\vert }}\left[\frac{\partial}{\partial q_{1}}\left(\sqrt{\left\vert g\right\vert }g^{11}\frac{\partial u}{\partial q_{1}}\right)+\frac{\partial}{\partial q_{2}}\left(\sqrt{\left\vert g\right\vert }g^{22}\frac{\partial u}{\partial q_{2}}\right)+\frac{\partial}{\partial q_{3}}\left(\sqrt{\left\vert g\right\vert }g^{33}\frac{\partial u}{\partial q_{3}}\right)\right]
\]

\[
\Delta u=\frac{1}{h_{1}h_{2}^{2}}\left[\frac{\partial}{\partial q_{1}}\left(\frac{h_{2}^{2}}{h_{1}}\frac{\partial u}{\partial q_{1}}\right)+\frac{\partial}{\partial q_{2}}\left(h_{1}\frac{\partial u}{\partial q_{2}}\right)+\frac{\partial}{\partial q_{3}}\left(h_{1}\frac{\partial u}{\partial q_{3}}\right)\right]
\]

Substituindo os valores

\[
\Delta u=\frac{1}{\sqrt{\left\vert g\right\vert }}\left[\frac{\partial}{\partial\theta_{1}}\left(\frac{a\,[senh^{2}(\chi)\,sen^{2}(\theta_{2})+cosh^{2}(\chi)\,cos^{2}(\theta_{2})]}{cosh(\chi)\,sen(\theta_{2})}\frac{\partial u}{\partial\theta_{1}}\right)+\frac{\partial}{\partial\theta_{2}}\left(a\,cosh(\chi)\,sen(\theta_{2})\frac{\partial u}{\partial\theta_{2}}\right)+\frac{\partial}{\partial\chi}\left(a\,cosh(\chi)\,sen(\theta_{2})\frac{\partial u}{\partial\chi}\right)\right]
\]

\[
\Delta u=\frac{1}{\sqrt{\left\vert g\right\vert }}\left[\left(\frac{a\,[senh^{2}(\chi)\,sen^{2}(\theta_{2})+cosh^{2}(\chi)\,cos^{2}(\theta_{2})]}{cosh(\chi)\,sen(\theta_{2})}\frac{\partial^{2}u}{\partial\theta_{1}^{2}}\right)+\frac{\partial}{\partial\theta_{2}}\left(a\,cosh(\chi)\,sen(\theta_{2})\frac{\partial u}{\partial\theta_{2}}\right)+\frac{\partial}{\partial\chi}\left(a\,cosh(\chi)\,sen(\theta_{2})\frac{\partial u}{\partial\chi}\right)\right]
\]

\[
\Delta u=\frac{1}{a\,cosh(\chi)\,sen(\theta_{2})\left\{ a^{2}\left[senh^{2}(\chi)sen^{2}(\theta_{2})+cosh^{2}(\chi)cos^{2}(\theta_{2})\right]\right\} }\left[\frac{\partial}{\partial\theta_{2}}\left(a\,cosh(\chi)\,sen(\theta_{2})\frac{\partial u}{\partial\theta_{2}}\right)+\frac{\partial}{\partial\chi}\left(a\,cosh(\chi)\,sen(\theta_{2})\frac{\partial u}{\partial\chi}\right)\right]
\]

\[
+\frac{1}{a^{2}\,cosh^{2}(\chi)\,sen^{2}(\theta_{2})}\frac{\partial^{2}u}{\partial\theta_{1}^{2}}
\]

\end{document}
