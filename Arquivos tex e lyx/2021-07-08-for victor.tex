
\documentclass[a4paper,12pt]{article}
%%%%%%%%%%%%%%%%%%%%%%%%%%%%%%%%%%%%%%%%%%%%%%%%%%%%%%%%%%%%%%%%%%%%%%%%%%%%%%%%%%%%%%%%%%%%%%%%%%%%%%%%%%%%%%%%%%%%%%%%%%%%%%%%%%%%%%%%%%%%%%%%%%%%%%%%%%%%%%%%%%%%%%%%%%%%%%%%%%%%%%%%%%%%%%%%%%%%%%%%%%%%%%%%%%%%%%%%%%%%%%%%%%%%%%%%%%%%%%%%%%%%%%%%%%%%
\usepackage{amssymb}
\usepackage{amsfonts}
\usepackage{graphicx}
\usepackage{amsmath}

\setcounter{MaxMatrixCols}{10}
%TCIDATA{OutputFilter=LATEX.DLL}
%TCIDATA{Version=5.50.0.2953}
%TCIDATA{<META NAME="SaveForMode" CONTENT="1">}
%TCIDATA{BibliographyScheme=Manual}
%TCIDATA{Created=Sat Oct 14 01:01:49 2006}
%TCIDATA{LastRevised=Thursday, July 08, 2021 23:04:31}
%TCIDATA{<META NAME="GraphicsSave" CONTENT="32">}
%TCIDATA{<META NAME="DocumentShell" CONTENT="Journal Articles\Standard LaTeX Article">}
%TCIDATA{CSTFile=LaTeX article (bright).cst}

\textheight=26.5cm
\textwidth=18cm
\oddsidemargin=-1.0cm
\evensidemargin=-1.0cm
\topmargin=-2.5cm
\def\baselinestretch{1.6}
\begin{document}


;

\section{operador de Laplace-Beltrami}

;

vamos aprender a t\'{e}cnica do operador de Laplace-Beltrami (LB)

;

Sec. 2.3 no [relAleff], p. 18 \{no arq tem erros de digita\c{c}\~{a}o\}

;

usando a parametriza\c{c}\~{a}o como est\'{a} no site;

so trocados%
\begin{equation*}
x_{1}\leftrightarrow x_{2}
\end{equation*}%
\begin{eqnarray*}
x_{3} &=&a\sinh \chi \sin \theta _{2} \\
x_{2} &=&a\cosh \chi \cos \theta _{2}\cos \theta _{1} \\
x_{1} &=&a\cosh \chi \cos \theta _{2}\sin \theta _{1} \\
0 &\leq &\chi <\infty ,\ -\pi /2\leq \theta _{2}\leq \pi /2,\ 0\leq \theta
_{1}<2\pi 
\end{eqnarray*}%
O operador de Laplace-Beltrami \'{e} definido como (implica soma por \'{\i}%
ndices repetidos):%
\begin{equation}
\Delta u=\frac{1}{\sqrt{\left\vert g\right\vert }}\frac{\partial }{\partial
q_{i}}\left( \sqrt{\left\vert g\right\vert }g^{ij}\frac{\partial u}{\partial
q_{j}}\right) \ ,  \label{eq98}
\end{equation}%
onde $g_{ij}$ \'{e} o tensor m\'{e}trico, $g^{ij}$ \'{e} o tensor m\'{e}%
trico inverso e$\ g$ \'{e} o determinante do tensor m\'{e}trico. O tensor m%
\'{e}trico pode ser determinado como [! aqui precisa colocar refer\^{e}%
ncia]: 
\begin{equation}
g_{ij}=\underset{k}{\sum }\frac{\partial x_{k}}{\partial q_{_{^{i}}}}\frac{%
\partial x_{^{_{k}}}}{\partial q_{_{^{j}}}}=G^{T}G\ .  \label{eq99}
\end{equation}%
No nosso caso:%
\begin{equation*}
q_{1}=\theta _{1},\ q_{2}=\theta _{2},\ q_{3}=\chi 
\end{equation*}%
Para determinar $g_{ij},$ $g,$ $g^{ij}$ tenho o programa na Maxima (em anexo
em .pdf), no programa $\theta _{1}\theta _{2}\theta _{3}$ denotadas por
t1,t2,t3 e $\chi $ por $v.$ Como coordenadas s\~{a}o ortogonais, o tensor m%
\'{e}trico \'{e} diagonal e pode ser escrito como%
\begin{equation*}
g_{ij}=diag\left( h_{1},h_{2},h_{3}\right) =diag\left(
h_{1},h_{2},h_{2}\right) 
\end{equation*}%
\begin{equation*}
h_{1}=a^{2}\cos ^{2}\theta _{2}\cosh ^{2}\chi ,\ h_{2}=h_{3}=a^{2}\left(
\cos ^{2}\theta _{2}\sinh ^{2}\chi +\sin ^{2}\theta _{2}\cosh ^{2}\chi
\right) 
\end{equation*}%
\begin{equation*}
g^{ij}=g_{ij}^{-1}=diag\left( h_{1}^{-1},h_{2}^{-1},h_{3}^{-1}\right)
=diag\left( h_{1}^{-1},h_{2}^{-1},h_{2}^{-1}\right) 
\end{equation*}%
\begin{equation*}
\left\vert g\right\vert =h_{1}h_{2}h_{3}=h_{1}h_{2}^{2}
\end{equation*}%
com $\partial _{i}=\frac{\partial }{\partial q_{i}}$%
\begin{equation*}
\frac{1}{\sqrt{\left\vert g\right\vert }}\frac{\partial }{\partial q_{i}}%
\left( \sqrt{\left\vert g\right\vert }g^{ij}\frac{\partial u}{\partial q_{j}}%
\right) =\frac{1}{\sqrt{\left\vert g\right\vert }}\partial _{i}\left( \sqrt{%
\left\vert g\right\vert }g^{ij}\partial _{j}u\right) =\frac{1}{\sqrt{%
\left\vert g\right\vert }}\left[ \partial _{i}\left( \sqrt{\left\vert
g\right\vert }g^{ij}\right) \right] \partial _{j}u+g^{ij}\partial
_{i}\partial _{j}u
\end{equation*}%
\begin{equation*}
\frac{1}{\sqrt{\left\vert g\right\vert }}\left[ \partial _{i}\left( \sqrt{%
\left\vert g\right\vert }g^{ij}\right) \right] \partial _{j}u=\frac{1}{%
2\left\vert g\right\vert }\left( \partial _{i}\left\vert g\right\vert
\right) g^{ij}\partial _{j}u+\left( \partial _{i}g^{ij}\right) \partial _{j}u
\end{equation*}%
aqui%
\begin{equation*}
g^{ij}\partial _{i}\partial _{j}u=h_{1}^{-1}\partial
_{1}^{2}u+h_{2}^{-1}\partial _{2}^{2}u+h_{3}^{-1}\partial
_{3}^{2}u=h_{1}^{-1}\partial _{1}^{2}u+h_{2}^{-1}\partial
_{2}^{2}u+h_{2}^{-1}\partial _{3}^{2}u=h_{1}^{-1}\partial
_{1}^{2}u+h_{2}^{-1}\left( \partial _{2}^{2}u+\partial _{3}^{2}u\right) 
\end{equation*}%
;%
\begin{equation*}
\left( \partial _{i}g^{ij}\right) \partial _{j}u=\left( \partial
_{1}h_{1}^{-1}\right) \left( \partial _{1}u\right) +\left( \partial
_{2}h_{2}^{-1}\right) \left( \partial _{2}u\right) +\left( \partial
_{3}h_{2}^{-1}\right) \left( \partial _{3}u\right) =\left( \partial
_{2}h_{2}^{-1}\right) \left( \partial _{2}u\right) +\left( \partial
_{3}h_{2}^{-1}\right) \left( \partial _{3}u\right) 
\end{equation*}%
pois $\left( \partial _{1}h_{1}^{-1}\right) =0$%
\begin{equation*}
=\left( -h_{2}^{-2}\right) \left( \partial _{2}h_{2}\right) \left( \partial
_{2}u\right) +\left( -h_{2}^{-2}\right) \left( \partial _{3}h_{2}\right)
\left( \partial _{3}u\right) 
\end{equation*}%
\begin{equation*}
=-\frac{h_{1}}{h_{1}h_{2}^{2}}\left[ \left( \partial _{2}h_{2}\right) \left(
\partial _{2}u\right) +\left( \partial _{3}h_{2}\right) \left( \partial
_{3}u\right) \right] 
\end{equation*}%
\begin{equation*}
=-\frac{h_{1}}{\left\vert g\right\vert }\left[ \left( \partial
_{2}h_{2}\right) \left( \partial _{2}u\right) +\left( \partial
_{3}h_{2}\right) \left( \partial _{3}u\right) \right] 
\end{equation*}%
pp%
\begin{equation*}
\left( \partial _{i}g^{ij}\right) \partial _{j}u=-\frac{h_{1}}{\left\vert
g\right\vert }\left[ \left( \partial _{2}h_{2}\right) \left( \partial
_{2}u\right) +\left( \partial _{3}h_{2}\right) \left( \partial _{3}u\right) %
\right] 
\end{equation*}%
;%
\begin{equation*}
\left( \partial _{i}\left\vert g\right\vert \right) g^{ij}\partial
_{j}u=\left( \partial _{1}\left\vert g\right\vert \right) h_{1}^{-1}\left(
\partial _{1}u\right) +\left( \partial _{2}\left\vert g\right\vert \right)
h_{2}^{-1}\left( \partial _{2}u\right) +\left( \partial _{3}\left\vert
g\right\vert \right) h_{2}^{-1}\left( \partial _{3}u\right) 
\end{equation*}%
\begin{equation*}
=\left( \partial _{2}\left\vert g\right\vert \right) h_{2}^{-1}\left(
\partial _{2}u\right) +\left( \partial _{3}\left\vert g\right\vert \right)
h_{2}^{-1}\left( \partial _{3}u\right) 
\end{equation*}%
pois $\left( \partial _{1}\left\vert g\right\vert \right) =0$ e 
\begin{equation*}
\partial _{i}\left\vert g\right\vert =\left( \partial _{i}h_{1}\right)
h_{2}^{2}+2h_{1}h_{2}\left( \partial _{i}h_{2}\right) =h_{2}\left(
h_{2}\left( \partial _{i}h_{1}\right) +2h_{1}\left( \partial
_{i}h_{2}\right) \right) =\frac{\left\vert g\right\vert \left( h_{2}\left(
\partial _{i}h_{1}\right) +2h_{1}\left( \partial _{i}h_{2}\right) \right) }{%
h_{1}h_{2}},\ i=2,3
\end{equation*}%
pp%
\begin{equation*}
\left( \partial _{i}\left\vert g\right\vert \right) g^{ij}\partial _{j}u=%
\frac{\left\vert g\right\vert \left( h_{2}\left( \partial _{2}h_{1}\right)
+2h_{1}\left( \partial _{2}h_{2}\right) \right) }{h_{1}h_{2}}%
h_{2}^{-1}\left( \partial _{2}u\right) +\frac{\left\vert g\right\vert \left(
h_{2}\left( \partial _{3}h_{1}\right) +2h_{1}\left( \partial
_{3}h_{2}\right) \right) }{h_{1}h_{2}}h_{2}^{-1}\left( \partial _{3}u\right) 
\end{equation*}%
\begin{equation*}
=\frac{\left\vert g\right\vert \left( h_{2}\left( \partial _{2}h_{1}\right)
+2h_{1}\left( \partial _{2}h_{2}\right) \right) }{h_{1}h_{2}^{2}}\left(
\partial _{2}u\right) +\frac{\left\vert g\right\vert \left( h_{2}\left(
\partial _{3}h_{1}\right) +2h_{1}\left( \partial _{3}h_{2}\right) \right) }{%
h_{1}h_{2}^{2}}\left( \partial _{3}u\right) 
\end{equation*}%
\begin{equation*}
=\left( h_{2}\left( \partial _{2}h_{1}\right) +2h_{1}\left( \partial
_{2}h_{2}\right) \right) \left( \partial _{2}u\right) +\left( h_{2}\left(
\partial _{3}h_{1}\right) +2h_{1}\left( \partial _{3}h_{2}\right) \right)
\left( \partial _{3}u\right) 
\end{equation*}%
;%
\begin{equation*}
\frac{1}{2\left\vert g\right\vert }\left( \partial _{i}\left\vert
g\right\vert \right) g^{ij}\partial _{j}u=\frac{1}{2\left\vert g\right\vert }%
\left[ \left( h_{2}\left( \partial _{2}h_{1}\right) +2h_{1}\left( \partial
_{2}h_{2}\right) \right) \left( \partial _{2}u\right) +\left( h_{2}\left(
\partial _{3}h_{1}\right) +2h_{1}\left( \partial _{3}h_{2}\right) \right)
\left( \partial _{3}u\right) \right] 
\end{equation*}%
ent\~{a}o%
\begin{equation*}
\frac{1}{2\left\vert g\right\vert }\left( \partial _{i}\left\vert
g\right\vert \right) g^{ij}\partial _{j}u=\frac{1}{2\left\vert g\right\vert }%
\left[ \left( h_{2}\left( \partial _{2}h_{1}\right) +2h_{1}\left( \partial
_{2}h_{2}\right) \right) \left( \partial _{2}u\right) +\left( h_{2}\left(
\partial _{3}h_{1}\right) +2h_{1}\left( \partial _{3}h_{2}\right) \right)
\left( \partial _{3}u\right) \right] 
\end{equation*}%
\begin{equation*}
\left( \partial _{i}g^{ij}\right) \partial _{j}u=-\frac{h_{1}}{\left\vert
g\right\vert }\left[ \left( \partial _{2}h_{2}\right) \left( \partial
_{2}u\right) +\left( \partial _{3}h_{2}\right) \left( \partial _{3}u\right) %
\right] 
\end{equation*}%
pp%
\begin{equation*}
\frac{1}{\sqrt{\left\vert g\right\vert }}\left[ \partial _{i}\left( \sqrt{%
\left\vert g\right\vert }g^{ij}\right) \right] \partial _{j}u=\frac{1}{%
2\left\vert g\right\vert }\left( \partial _{i}\left\vert g\right\vert
\right) g^{ij}\partial _{j}u+\left( \partial _{i}g^{ij}\right) \partial _{j}u
\end{equation*}%
\begin{eqnarray*}
&=&\frac{1}{2\left\vert g\right\vert }\left[ \left( h_{2}\left( \partial
_{2}h_{1}\right) +2h_{1}\left( \partial _{2}h_{2}\right) \right) \left(
\partial _{2}u\right) +\left( h_{2}\left( \partial _{3}h_{1}\right)
+2h_{1}\left( \partial _{3}h_{2}\right) \right) \left( \partial _{3}u\right) %
\right]  \\
&&-\frac{h_{1}}{\left\vert g\right\vert }\left[ \left( \partial
_{2}h_{2}\right) \left( \partial _{2}u\right) +\left( \partial
_{3}h_{2}\right) \left( \partial _{3}u\right) \right] 
\end{eqnarray*}%
\begin{eqnarray*}
&=&\frac{1}{2\left\vert g\right\vert }\left[ \left( h_{2}\left( \partial
_{2}h_{1}\right) +2h_{1}\left( \partial _{2}h_{2}\right) \right)
-2h_{1}\left( \partial _{2}h_{2}\right) \right] \left( \partial _{2}u\right) 
\\
&&+\frac{1}{2\left\vert g\right\vert }\left[ \left( h_{2}\left( \partial
_{3}h_{1}\right) +2h_{1}\left( \partial _{3}h_{2}\right) \right)
-2h_{1}\left( \partial _{3}h_{2}\right) \right] \left( \partial _{3}u\right) 
\end{eqnarray*}%
\begin{equation*}
=\frac{h_{2}}{2\left\vert g\right\vert }\left[ \left( \partial
_{2}h_{1}\right) \left( \partial _{2}u\right) +\left( \partial
_{3}h_{1}\right) \left( \partial _{3}u\right) \right] 
\end{equation*}%
ent\~{a}o%
\begin{equation*}
\frac{1}{\sqrt{\left\vert g\right\vert }}\left[ \partial _{i}\left( \sqrt{%
\left\vert g\right\vert }g^{ij}\right) \right] \partial _{j}u=\frac{h_{2}}{%
2\left\vert g\right\vert }\left[ \left( \partial _{2}h_{1}\right) \left(
\partial _{2}u\right) +\left( \partial _{3}h_{1}\right) \left( \partial
_{3}u\right) \right] 
\end{equation*}%
e%
\begin{equation*}
\left( \partial _{2}h_{1}\right) =-2a^{2}\cos \theta _{2}\cosh ^{2}\chi \sin
\theta _{2}=-2h_{1}\frac{\sin \theta _{2}}{\cos \theta _{2}}=-2h_{1}\tan
\theta _{2}
\end{equation*}%
\begin{equation*}
\left( \partial _{3}h_{1}\right) =2a^{2}\cos ^{2}\theta _{2}\cosh \chi \sinh
\chi =2h_{1}\frac{\sinh \chi }{\cosh \chi }=2h_{1}\tanh \chi 
\end{equation*}%
pp%
\begin{equation*}
\frac{1}{\sqrt{\left\vert g\right\vert }}\left[ \partial _{i}\left( \sqrt{%
\left\vert g\right\vert }g^{ij}\right) \right] \partial _{j}u=\frac{h_{2}}{%
2\left\vert g\right\vert }\left[ -2h_{1}\tan \theta _{2}\left( \partial
_{2}u\right) +2h_{1}\tanh \chi \left( \partial _{3}u\right) \right] 
\end{equation*}%
\begin{equation*}
=\frac{2h_{2}h_{1}}{2h_{1}h_{2}^{2}}\left[ -\tan \theta _{2}\left( \partial
_{2}u\right) +\tanh \chi \left( \partial _{3}u\right) \right] 
\end{equation*}%
\begin{equation*}
=\frac{1}{h_{2}}\left[ -\tan \theta _{2}\left( \partial _{2}u\right) +\tanh
\chi \left( \partial _{3}u\right) \right] 
\end{equation*}%
pp%
\begin{equation*}
\frac{1}{\sqrt{\left\vert g\right\vert }}\partial _{i}\left( \sqrt{%
\left\vert g\right\vert }g^{ij}\partial _{j}u\right) =\frac{1}{\sqrt{%
\left\vert g\right\vert }}\left[ \partial _{i}\left( \sqrt{\left\vert
g\right\vert }g^{ij}\right) \right] \partial _{j}u+g^{ij}\partial
_{i}\partial _{j}u
\end{equation*}%
\begin{equation*}
=\frac{1}{h_{2}}\left[ -\tan \theta _{2}\left( \partial _{2}u\right) +\tanh
\chi \left( \partial _{3}u\right) \right] +h_{1}^{-1}\partial
_{1}^{2}u+h_{2}^{-1}\left( \partial _{2}^{2}u+\partial _{3}^{2}u\right) 
\end{equation*}%
\begin{equation*}
=\frac{1}{h_{2}}\left[ -\tan \theta _{2}\left( \partial _{2}u\right) +\tanh
\chi \left( \partial _{3}u\right) +\partial _{2}^{2}u+\partial _{3}^{2}u%
\right] +\frac{1}{h_{1}}\partial _{1}^{2}u
\end{equation*}%
\begin{equation*}
=\frac{1}{h_{2}}\left[ -\tan \theta _{2}\left( \partial _{2}u\right)
+\partial _{2}^{2}u+\tanh \chi \left( \partial _{3}u\right) +\partial
_{3}^{2}u\right] +\frac{1}{h_{1}}\partial _{1}^{2}u
\end{equation*}%
aqui%
\begin{equation*}
\left[ -\tan \theta _{2}\left( \partial _{2}u\right) +\partial
_{2}^{2}u+\tanh \chi \left( \partial _{3}u\right) +\left( \partial
_{3}^{2}u\right) \right] =\frac{1}{\cos \theta _{2}}\partial _{2}\left( \cos
\theta _{2}\partial _{2}u\right) +\frac{1}{\cosh \chi }\partial _{3}\left(
\cosh \chi \partial _{3}u\right) 
\end{equation*}%
pp%
\begin{equation*}
\frac{1}{\sqrt{\left\vert g\right\vert }}\partial _{i}\left( \sqrt{%
\left\vert g\right\vert }g^{ij}\partial _{j}u\right) =
\end{equation*}%
\begin{equation*}
=\frac{1}{h_{2}}\left[ \frac{1}{\cos \theta _{2}}\partial _{2}\left( \cos
\theta _{2}\partial _{2}u\right) +\frac{1}{\cosh \chi }\partial _{3}\left(
\cosh \chi \partial _{3}u\right) \right] +\frac{1}{h_{1}}\partial _{1}^{2}u
\end{equation*}%
aplicando 
\begin{equation*}
h_{1}=a^{2}\cos ^{2}\theta _{2}\cosh ^{2}\chi ,\ h_{2}=h_{3}=a^{2}\left(
\cos ^{2}\theta _{2}\sinh ^{2}\chi +\sin ^{2}\theta _{2}\cosh ^{2}\chi
\right) 
\end{equation*}%
obtemos%
\begin{equation*}
\frac{1}{\sqrt{\left\vert g\right\vert }}\partial _{i}\left( \sqrt{%
\left\vert g\right\vert }g^{ij}\partial _{j}u\right) =
\end{equation*}%
\begin{eqnarray*}
&=&\frac{1}{a^{2}\left( \cos ^{2}\theta _{2}\sinh ^{2}\chi +\sin ^{2}\theta
_{2}\cosh ^{2}\chi \right) }\left[ \frac{1}{\cos \theta _{2}}\partial
_{2}\left( \cos \theta _{2}\partial _{2}u\right) +\frac{1}{\cosh \chi }%
\partial _{3}\left( \cosh \chi \partial _{3}u\right) \right]  \\
&&+\frac{1}{a^{2}\cos ^{2}\theta _{2}\cosh ^{2}\chi }\partial _{1}^{2}u
\end{eqnarray*}%
observamos tamb\'{e}m que 
\begin{equation*}
\left( \cos ^{2}\theta _{2}\sinh ^{2}\chi +\sin ^{2}\theta _{2}\cosh
^{2}\chi \right) =\left( 1-\sin ^{2}\theta _{2}\right) \sinh ^{2}\chi +\sin
^{2}\theta _{2}\cosh ^{2}\chi 
\end{equation*}%
\begin{equation*}
=\sinh ^{2}\chi +\sin ^{2}\theta _{2}\left( \cosh ^{2}\chi -\sinh ^{2}\chi
\right) =\sinh ^{2}\chi +\sin ^{2}\theta _{2}
\end{equation*}%
e podemos escrever%
\begin{equation*}
\frac{1}{\sqrt{\left\vert g\right\vert }}\partial _{i}\left( \sqrt{%
\left\vert g\right\vert }g^{ij}\partial _{j}u\right) =
\end{equation*}%
\begin{eqnarray*}
&=&\frac{1}{a^{2}\left( \sinh ^{2}\chi +\sin ^{2}\theta _{2}\right) }\left[ 
\frac{1}{\cos \theta _{2}}\partial _{2}\left( \cos \theta _{2}\partial
_{2}u\right) +\frac{1}{\cosh \chi }\partial _{3}\left( \cosh \chi \partial
_{3}u\right) \right]  \\
&&+\frac{1}{a^{2}\cos ^{2}\theta _{2}\cosh ^{2}\chi }\partial _{1}^{2}u
\end{eqnarray*}%
Finalmente obtemos o Laplaciano na coordenadas esferoidais:%
\begin{eqnarray*}
\nabla ^{2}u &=&\frac{1}{a^{2}\left( \sinh ^{2}\chi +\sin ^{2}\theta
_{2}\right) }\left[ \frac{1}{\cos \theta _{2}}\partial _{2}\left( \cos
\theta _{2}\partial _{2}u\right) +\frac{1}{\cosh \chi }\partial _{3}\left(
\cosh \chi \partial _{3}u\right) \right]  \\
&&+\frac{1}{a^{2}\cos ^{2}\theta _{2}\cosh ^{2}\chi }\partial _{1}^{2}u
\end{eqnarray*}%
ou 
\begin{eqnarray*}
\nabla ^{2}u &=&\frac{1}{a^{2}\left( \sinh ^{2}\chi +\sin ^{2}\theta
_{2}\right) }\left[ \frac{1}{\cos \theta _{2}}\frac{\partial }{\partial
\theta _{2}}\left( \cos \theta _{2}\frac{\partial u}{\partial \theta _{2}}%
\right) +\frac{1}{\cosh \chi }\frac{\partial }{\partial \chi }\left( \cosh
\chi \frac{\partial u}{\partial \chi }\right) \right]  \\
&&+\frac{1}{a^{2}\cos ^{2}\theta _{2}\cosh ^{2}\chi }\frac{\partial ^{2}u}{%
\partial \theta _{1}^{2}}
\end{eqnarray*}%
;

\section{proxima etapa}

;

obter Laplaciano na parametriza\c{c}\~{a}o%
\begin{eqnarray*}
x_{3} &=&a\sinh \chi \cos \theta _{2} \\
x_{2} &=&a\cosh \chi \sin \theta _{2}\cos \theta _{1} \\
x_{1} &=&a\cosh \chi \sin \theta _{2}\sin \theta _{1} \\
0 &\leq &\chi <\infty ,\ 0\leq \theta _{2}\leq \pi ,\ 0\leq \theta _{1}<2\pi 
\end{eqnarray*}%
e depois passar para dimens\~{a}o 4.

Na dimens\~{a}o 4 usamos as coordenadas:%
\begin{eqnarray*}
x_{4} &=&a\sinh \chi \cos \theta _{3} \\
x_{3} &=&a\cosh \chi \sin \theta _{3}\cos \theta _{2} \\
x_{1} &=&a\cosh \chi \sin \theta _{3}\sin \theta _{2}\cos \theta _{1} \\
x_{2} &=&a\cosh \chi \sin \theta _{3}\sin \theta _{2}\sin \theta _{1} \\
0 &\leq &\chi <\infty ,\ 0\leq \theta _{3},\theta _{2}\leq \pi ,\ 0\leq
\theta _{1}<2\pi 
\end{eqnarray*}%
por motivo da simetria na parametriza\c{c}\~{a}o \'{e} trocado $%
x_{1}\leftrightarrow x_{2},$ facilita os calculos (mas nesse caso surge uma
quest\~{a}o de orienta\c{c}\~{a}o de eixos de coordenadas; isso n\~{a}o \'{e}
dr\'{a}stico, mas guardamos isso na mente).

ent\~{a}o, usaremos coordenadas:%
\begin{eqnarray*}
x_{4} &=&a\sinh \chi \cos \theta _{3} \\
x_{3} &=&a\cosh \chi \sin \theta _{3}\cos \theta _{2} \\
x_{2} &=&a\cosh \chi \sin \theta _{3}\sin \theta _{2}\cos \theta _{1} \\
x_{1} &=&a\cosh \chi \sin \theta _{3}\sin \theta _{2}\sin \theta _{1} \\
0 &\leq &\chi <\infty ,\ 0\leq \theta _{3},\theta _{2}\leq \pi ,\ 0\leq
\theta _{1}<2\pi 
\end{eqnarray*}%
;

\end{document}
