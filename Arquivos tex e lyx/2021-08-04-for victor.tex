
\documentclass[a4paper,12pt]{article}
%%%%%%%%%%%%%%%%%%%%%%%%%%%%%%%%%%%%%%%%%%%%%%%%%%%%%%%%%%%%%%%%%%%%%%%%%%%%%%%%%%%%%%%%%%%%%%%%%%%%%%%%%%%%%%%%%%%%%%%%%%%%%%%%%%%%%%%%%%%%%%%%%%%%%%%%%%%%%%%%%%%%%%%%%%%%%%%%%%%%%%%%%%%%%%%%%%%%%%%%%%%%%%%%%%%%%%%%%%%%%%%%%%%%%%%%%%%%%%%%%%%%%%%%%%%%
\usepackage{amssymb}
\usepackage{amsfonts}
\usepackage{graphicx}
\usepackage{amsmath}

\setcounter{MaxMatrixCols}{10}
%TCIDATA{OutputFilter=LATEX.DLL}
%TCIDATA{Version=5.50.0.2953}
%TCIDATA{<META NAME="SaveForMode" CONTENT="1">}
%TCIDATA{BibliographyScheme=Manual}
%TCIDATA{Created=Sat Oct 14 01:01:49 2006}
%TCIDATA{LastRevised=Wednesday, August 04, 2021 14:12:42}
%TCIDATA{<META NAME="GraphicsSave" CONTENT="32">}
%TCIDATA{<META NAME="DocumentShell" CONTENT="Journal Articles\Standard LaTeX Article">}
%TCIDATA{CSTFile=LaTeX article (bright).cst}

\textheight=26.5cm
\textwidth=18cm
\oddsidemargin=-1.0cm
\evensidemargin=-1.0cm
\topmargin=-2.5cm
\def\baselinestretch{1.6}
\input{tcilatex}
\begin{document}


;

vc pode fazer para dim 4

; 

\section{Laplaciano na dim 4}

;

o Laplaciano j\'{a} \'{e} obtido

sim arrumar os termos na forma apresenada embaixo

vc deve perceber que nas coordenadas esferoidais achatadas%
\begin{eqnarray*}
\nabla ^{2}u &=&\frac{1}{a^{2}\left( \sinh ^{2}\chi +\cos ^{2}\theta
_{3}\right) }\left( \frac{1}{\sin ^{2}\theta _{3}}\frac{\partial }{\partial
\theta _{3}}\left( \sin ^{2}\theta _{3}\frac{\partial u}{\partial \theta _{3}%
}\right) +\frac{1}{\cosh ^{2}\chi }\frac{\partial }{\partial \chi }\left(
\cosh ^{2}\chi \frac{\partial u}{\partial \chi }\right) \right)  \\
&&+\frac{1}{a^{2}\sin ^{2}\theta _{3}\cosh ^{2}\chi }\nabla _{S^{2}}^{2}u\
\left( \dim \ 4\right) 
\end{eqnarray*}%
onde 
\begin{equation*}
\nabla _{S^{2}}^{2}u=\left[ \left( \frac{1}{\sin \theta _{2}}\frac{\partial 
}{\partial \theta _{2}}\left( \sin \theta _{2}\frac{\partial u}{\partial
\theta _{2}}\right) \right) +\frac{1}{\sin ^{2}\theta _{2}}\frac{\partial
^{2}u}{\partial \theta _{1}^{2}}\right] 
\end{equation*}%
\'{e} o Laplaciano na esfera $S^{2}$

;

neste caso de 4 dim, deve repretir os passos da separa\c{c}\~{a}o de vari%
\'{a}veis de 3 dim.

o problema 

\begin{equation*}
\nabla _{S^{2}}^{2}u=-\lambda _{2}u
\end{equation*}%
\'{e} resolvida ja. solu\c{c}\~{o}es s\~{a}o harm\^{o}nicos esf\'{e}ricos
que foram discutidos na fm2. 

deve chegar as equa\c{c}\~{o}es para $\Theta _{3}\left( \theta _{3}\right) $
e para $X\left( \chi \right) $

neste caso a eq para $\Theta _{3}\left( \theta _{3}\right) $ ser\'{a} a eq
de Gegenbauer. as solu\c{c}\~{o}es n\~{a}o singulares ser\~{a}o os
polinomios de Gegenbauer.

mas primeiramente chega as equa\c{c}\~{o}es!%
\begin{equation*}
\end{equation*}%
;

\end{document}
