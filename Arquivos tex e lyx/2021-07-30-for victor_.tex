
\documentclass[a4paper,12pt]{article}
%%%%%%%%%%%%%%%%%%%%%%%%%%%%%%%%%%%%%%%%%%%%%%%%%%%%%%%%%%%%%%%%%%%%%%%%%%%%%%%%%%%%%%%%%%%%%%%%%%%%%%%%%%%%%%%%%%%%%%%%%%%%%%%%%%%%%%%%%%%%%%%%%%%%%%%%%%%%%%%%%%%%%%%%%%%%%%%%%%%%%%%%%%%%%%%%%%%%%%%%%%%%%%%%%%%%%%%%%%%%%%%%%%%%%%%%%%%%%%%%%%%%%%%%%%%%
\usepackage{amssymb}
\usepackage{amsfonts}
\usepackage{graphicx}
\usepackage{amsmath}

\setcounter{MaxMatrixCols}{10}
%TCIDATA{OutputFilter=LATEX.DLL}
%TCIDATA{Version=5.50.0.2953}
%TCIDATA{<META NAME="SaveForMode" CONTENT="1">}
%TCIDATA{BibliographyScheme=Manual}
%TCIDATA{Created=Sat Oct 14 01:01:49 2006}
%TCIDATA{LastRevised=Friday, July 30, 2021 13:23:09}
%TCIDATA{<META NAME="GraphicsSave" CONTENT="32">}
%TCIDATA{<META NAME="DocumentShell" CONTENT="Journal Articles\Standard LaTeX Article">}
%TCIDATA{CSTFile=LaTeX article (bright).cst}

\textheight=26.5cm
\textwidth=18cm
\oddsidemargin=-1.0cm
\evensidemargin=-1.0cm
\topmargin=-2.5cm
\def\baselinestretch{1.6}
\input{tcilatex}
\begin{document}


\section{caso de 3 dim}

;%
\begin{eqnarray*}
x_{3} &=&a\sinh \chi \cos \theta _{2} \\
x_{2} &=&a\cosh \chi \sin \theta _{2}\cos \theta _{1} \\
x_{1} &=&a\cosh \chi \sin \theta _{2}\sin \theta _{1} \\
0 &\leq &\chi <\infty ,\ 0\leq \theta _{2}\leq \pi ,\ 0\leq \theta _{1}<2\pi
\end{eqnarray*}

\begin{eqnarray*}
\nabla ^{2}u &=&\frac{1}{a^{2}\left( \sinh ^{2}\chi +\cos ^{2}\theta
_{2}\right) }\left[ \frac{1}{\sin \theta _{2}}\frac{\partial }{\partial
\theta _{2}}\left( \sin \theta _{2}\frac{\partial u}{\partial \theta _{2}}%
\right) +\frac{1}{\cosh \chi }\frac{\partial }{\partial \chi }\left( \cosh
\chi \frac{\partial u}{\partial \chi }\right) \right] \\
&&+\frac{1}{a^{2}\sin ^{2}\theta _{2}\cosh ^{2}\chi }\frac{\partial ^{2}u}{%
\partial \theta _{1}^{2}}=0
\end{eqnarray*}%
aplica separa\c{c}\~{a}o de vari\'{a}veis na eq.

1ro separar a variavel $\theta _{1}$ com a constante de separa\c{c}\~{a}o $%
\lambda _{1}$

depois separar as vari\'{a}veis $\theta _{2}$ e $\chi $ a constante de separa%
\c{c}\~{a}o $\lambda _{2}$

deve obter 3 equa\c{c}\~{o}es dif. ordinarias com constantes de separa\c{c}%
\~{a}o $\lambda _{1}$ e $\lambda _{2}$

aplicar as condi\c{c}\~{o}es de contorno: 
\begin{equation*}
u\left( \theta _{1},\theta _{2},\chi \right) =u\left( \theta _{1}+2\pi
,\theta _{2},\chi \right)
\end{equation*}%
periodicidade (que segue da univocacidade) pela vari\'{a}vel $\theta _{1}$

e a condi\c{c}\~{a}o que solu\c{c}\~{a}o deve ser n\~{a}o singular dentro de
intervalos de varia\c{c}\~{a}o das vari\'{a}veis

;%
\begin{equation*}
u(\theta _{1},\theta _{2},\chi )=\Theta _{1}(\theta _{1})W(\theta _{2},\chi )
\end{equation*}%
melhor deixar na forma:%
\begin{equation*}
-\frac{\partial ^{2}\Theta _{1}}{\partial \theta _{1}^{2}}\frac{1}{\Theta
_{1}}=\frac{1}{W}\frac{\sin ^{2}\theta _{2}\cosh ^{2}\chi }{\left( \sinh
^{2}\chi +\cos ^{2}\theta _{2}\right) }\left[ \frac{1}{\sin \theta _{2}}%
\frac{\partial }{\partial \theta _{2}}\left( \sin \theta _{2}\frac{\partial W%
}{\partial \theta _{2}}\right) +\frac{1}{\cosh \chi }\frac{\partial }{%
\partial \chi }\left( \cosh \chi \frac{\partial W}{\partial \chi }\right) %
\right] =\lambda _{1}
\end{equation*}%
\begin{equation*}
-\frac{\partial ^{2}\Theta _{1}}{\partial \theta _{1}^{2}}\frac{1}{\Theta
_{1}}=\lambda _{1}\Rightarrow \frac{d^{2}\Theta _{1}}{d\theta _{1}^{2}}%
+\lambda _{1}\Theta _{1}=0
\end{equation*}%
a 1ra eq dif ordinaria com condi\c{c}\~{a}o de contorno (cc) \'{e}%
\begin{equation*}
\frac{d^{2}\Theta _{1}}{d\theta _{1}^{2}}+\lambda _{1}\Theta _{1}=0,\ \Theta
_{1}\left( \theta _{1}\right) =\Theta _{1}\left( \theta _{1}+2\pi \right)
\end{equation*}%
resolve a eq (para vc escrever) obtem que 
\begin{equation*}
\lambda _{1}=m^{2},\ m\in \mathbb{Z}
\end{equation*}%
\begin{equation*}
\Theta _{1}\left( \theta _{1}\right) =Ae^{im\theta _{1}}+Be^{-im\theta _{1}}
\end{equation*}%
pr\'{o}ximo passo%
\begin{equation*}
W(\theta _{2},\chi )=\Theta _{2}(\theta _{2})X(\chi )
\end{equation*}%
\begin{equation*}
\left[ \frac{1}{\sin \theta _{2}}\frac{\partial }{\partial \theta _{2}}%
\left( \sin \theta _{2}\frac{\partial W}{\partial \theta _{2}}\right) +\frac{%
1}{\cosh \chi }\frac{\partial }{\partial \chi }\left( \cosh \chi \frac{%
\partial W}{\partial \chi }\right) \right] =\lambda _{1}W\frac{\left( \sinh
^{2}\chi +\cos ^{2}\theta _{2}\right) }{\sin ^{2}\theta _{2}\cosh ^{2}\chi }
\end{equation*}%
aqui 
\begin{equation*}
\frac{\left( \sinh ^{2}\chi +\cos ^{2}\theta _{2}\right) }{\sin ^{2}\theta
_{2}\cosh ^{2}\chi }=\frac{1}{\sin ^{2}\theta _{2}}+\frac{1}{\cosh ^{2}\chi }
\end{equation*}%
p%
\begin{equation*}
\left[ X\frac{1}{\sin \theta _{2}}\frac{\partial }{\partial \theta _{2}}%
\left( \sin \theta _{2}\frac{\partial \Theta _{2}}{\partial \theta _{2}}%
\right) +\Theta _{2}\frac{1}{\cosh \chi }\frac{\partial }{\partial \chi }%
\left( \cosh \chi \frac{\partial X}{\partial \chi }\right) \right] =\lambda
_{1}\Theta _{2}X\left( \frac{1}{\sin ^{2}\theta _{2}}+\frac{1}{\cosh
^{2}\chi }\right)
\end{equation*}%
\begin{equation*}
\left[ \frac{1}{\Theta _{2}}\frac{1}{\sin \theta _{2}}\frac{\partial }{%
\partial \theta _{2}}\left( \sin \theta _{2}\frac{\partial \Theta _{2}}{%
\partial \theta _{2}}\right) +\frac{1}{X}\frac{1}{\cosh \chi }\frac{\partial 
}{\partial \chi }\left( \cosh \chi \frac{\partial X}{\partial \chi }\right) %
\right] =\lambda _{1}\left( \frac{1}{\sin ^{2}\theta _{2}}+\frac{1}{\cosh
^{2}\chi }\right)
\end{equation*}%
\begin{equation*}
-\frac{1}{\Theta _{2}}\frac{1}{\sin \theta _{2}}\frac{\partial }{\partial
\theta _{2}}\left( \sin \theta _{2}\frac{\partial \Theta _{2}}{\partial
\theta _{2}}\right) +\lambda _{1}\frac{1}{\sin ^{2}\theta _{2}}=\frac{1}{X}%
\frac{1}{\cosh \chi }\frac{\partial }{\partial \chi }\left( \cosh \chi \frac{%
\partial X}{\partial \chi }\right) -\lambda _{1}\frac{1}{\cosh ^{2}\chi }%
=\lambda _{2}
\end{equation*}%
portanto%
\begin{equation*}
-\frac{1}{\Theta _{2}}\frac{1}{\sin \theta _{2}}\frac{\partial }{\partial
\theta _{2}}\left( \sin \theta _{2}\frac{\partial \Theta _{2}}{\partial
\theta _{2}}\right) +\lambda _{1}\frac{1}{\sin ^{2}\theta _{2}}=\lambda _{2}
\end{equation*}%
\begin{equation*}
\frac{1}{X}\frac{1}{\cosh \chi }\frac{\partial }{\partial \chi }\left( \cosh
\chi \frac{\partial X}{\partial \chi }\right) -\lambda _{1}\frac{1}{\cosh
^{2}\chi }=\lambda _{2}
\end{equation*}%
ou%
\begin{equation*}
\frac{1}{\sin \theta _{2}}\frac{\partial }{\partial \theta _{2}}\left( \sin
\theta _{2}\frac{\partial \Theta _{2}}{\partial \theta _{2}}\right) +\left(
\lambda _{2}-\lambda _{1}\frac{1}{\sin ^{2}\theta _{2}}\right) \Theta _{2}=0
\end{equation*}%
\begin{equation*}
\frac{1}{\cosh \chi }\frac{\partial }{\partial \chi }\left( \cosh \chi \frac{%
\partial X}{\partial \chi }\right) -\left( \lambda _{2}+\lambda _{1}\frac{1}{%
\cosh ^{2}\chi }\right) X=0
\end{equation*}%
podemos aplicar $\lambda _{1}=m^{2}$%
\begin{equation*}
\frac{1}{\sin \theta _{2}}\frac{\partial }{\partial \theta _{2}}\left( \sin
\theta _{2}\frac{\partial \Theta _{2}}{\partial \theta _{2}}\right) +\left(
\lambda _{2}-\frac{m^{2}}{\sin ^{2}\theta _{2}}\right) \Theta _{2}=0
\end{equation*}%
\begin{equation*}
\frac{1}{\cosh \chi }\frac{\partial }{\partial \chi }\left( \cosh \chi \frac{%
\partial X}{\partial \chi }\right) -\left( \lambda _{2}+\frac{m^{2}}{\cosh
^{2}\chi }\right) X=0
\end{equation*}%
Na eq para $\Theta _{2}$ vc reconhece a eq de Legendre. Solu\c{c}\~{o}es n%
\~{a}o singulares expressas em termos de polinomios de Legendre. E
determinamos $\lambda _{2}=n\left( n+1\right) ,$ $n\in \mathbb{N}%
_{0}=\left\{ 0,1,2,3,...\right\} .$ Escreva as solu\c{c}\~{o}es!

$\frac{{}}{{}}$

Na eq para $X$ aplica a vari\'{a}vel 
\begin{equation*}
\zeta =\sinh \chi ,\ 0\leq \zeta <\infty 
\end{equation*}%
e chega a eq na var $\zeta $ : escreva a eq !%
\begin{equation*}
\end{equation*}

\end{document}
