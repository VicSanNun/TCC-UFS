\documentclass[12pt, a4papper]{article}
\usepackage[utf8]{inputenc}
\usepackage[brazil]{babel}
\usepackage{amsmath}
\usepackage{amsfonts}
\usepackage{amssymb}
\usepackage{verbatim}
\usepackage[T1]{fontenc}
\usepackage{graphicx}
\usepackage{physics}
\usepackage{siunitx}
\usepackage[top=1cm, bottom=1cm, left=1cm, right=1cm]{geometry}
\usepackage{makeidx}
\usepackage[utf8]{inputenc}
\newtheorem{defi}{Defini��o}[subsection]
\newtheorem{teor}{Teorema}[subsection]
\newtheorem{demo}{Demonstra��o}[subsection]
\newtheorem{nota}{Nota��o}[subsection]
\newtheorem{coro}{Corol�rio}[subsection]

\begin{document}
De modo geral, para verificar a ortogonalidade do sistema de coordenadas
� necess�rio verificar se os vetores da base do sistema ($\boldsymbol{e}_{i}$)
s�o ortogonais. Para isso, basta verificar se o produto escalar, entre
dois vetores da base, � nulo ou seja:

\[
\boldsymbol{e}_{i}\cdot\boldsymbol{e}_{j}=0
\]

Por�m, antes disso, � necess�rio encontrar os vetores da base. Para
isto, basta tomar $\boldsymbol{r}(q_{1},q_{2},q_{3},...,q_{n})$ e
tem-se o seguinte:

\[
\boldsymbol{e}_{i}=\frac{\partial\boldsymbol{r}}{\partial q_{i}}
\]


\section*{Verificando Ortogonalidade do Hiperbol�ide 3-dim (3-Hiperboloide?)
de 1 Folha}

Verificar a ortogonalidade do seguinte sistema de coordenadas

\begin{eqnarray*}
x_{3} & = & a\sinh\chi\cos\theta_{2}\\
x_{2} & = & a\cosh\chi\sin\theta_{2}\cos\theta_{1}\\
x_{1} & = & a\cosh\chi\sin\theta_{2}\sin\theta_{1}\\
0 & \leq & \chi<\infty,\ 0\leq\theta_{2}\leq\pi,\ 0\leq\theta_{1}<2\pi
\end{eqnarray*}

Define-se

\[
\boldsymbol{r=}(a\cosh\chi\sin\theta_{2}\sin\theta_{1})\boldsymbol{\widehat{i}}+(a\cosh\chi\sin\theta_{2}\cos\theta_{1})\boldsymbol{\widehat{j}}+(a\sinh\chi\cos\theta_{2})\boldsymbol{\widehat{k}}
\]

Tem-se

\[
\boldsymbol{e}_{\theta_{1}}=\frac{\partial\boldsymbol{r}}{\partial\theta_{1}}=\left(\frac{\partial}{\partial\theta_{1}}(a\cosh\chi\sin\theta_{2}\sin\theta_{1})\boldsymbol{\widehat{i}}+\frac{\partial}{\partial\theta_{1}}(a\cosh\chi\sin\theta_{2}\cos\theta_{1})\boldsymbol{\widehat{j}}+\frac{\partial}{\partial\theta_{1}}(a\sinh\chi\cos\theta_{2})\boldsymbol{\widehat{k}}\right)
\]

\[
=\left(a\,cosh(\chi)cos(\theta_{1})sin(\theta_{2})\boldsymbol{\widehat{i}}-a\,cosh(\chi)sin(\theta_{1})sin(\theta_{2})\boldsymbol{\widehat{j}}+0\boldsymbol{\widehat{k}}\right)
\]

\[
\boldsymbol{e}_{\theta_{2}}=\frac{\partial\boldsymbol{r}}{\partial\theta_{2}}=\left(\frac{\partial}{\partial\theta_{2}}(a\cosh\chi\sin\theta_{2}\sin\theta_{1})\boldsymbol{\widehat{i}}+\frac{\partial}{\partial\theta_{2}}(a\cosh\chi\sin\theta_{2}\cos\theta_{1})\boldsymbol{\widehat{j}}+\frac{\partial}{\partial\theta_{2}}(a\sinh\chi\cos\theta_{2})\boldsymbol{\widehat{k}}\right)
\]

\[
=\left(a\,cosh(\chi)sin(\theta_{1})cos(\theta_{2})\boldsymbol{\widehat{i}}+a\,cosh(\chi)cos(\theta_{1})cos(\theta_{2})\boldsymbol{\widehat{j}}-a\,sinh(\chi)sin(\theta_{2})\boldsymbol{\widehat{k}}\right)
\]

\[
\boldsymbol{e}_{\chi}=\frac{\partial\boldsymbol{r}}{\partial\chi}=\left(\frac{\partial}{\partial\chi}(a\cosh\chi\sin\theta_{2}\sin\theta_{1})\boldsymbol{\widehat{i}}+\frac{\partial}{\partial\chi}(a\cosh\chi\sin\theta_{2}\cos\theta_{1})\boldsymbol{\widehat{j}}+\frac{\partial}{\partial\chi}(a\sinh\chi\cos\theta_{2})\boldsymbol{\widehat{k}}\right)
\]

\[
=\left(a\,sinh(\chi)sin(\theta_{1})sin(\theta_{2})\boldsymbol{\widehat{i}}+a\,sinh(\chi)cos(\theta_{1})sin(\theta_{2})\boldsymbol{\widehat{j}}+a\,cosh(\chi)cos(\theta_{2})\boldsymbol{\widehat{k}}\right)
\]
\\
Agora basta verificar a ortogonalidade dois a dois entre estes vetores.

\[
\boldsymbol{e}_{\theta_{1}}\cdot\boldsymbol{e}_{\theta_{2}}=\left(a^{2}\,cosh^{2}(\chi)cos(\theta_{1})sin(\theta_{2})sin(\theta_{1})cos(\theta_{2})-a^{2}\,cosh^{2}(\chi)sin(\theta_{1})sin(\theta_{2})cos(\theta_{1})cos(\theta_{2})\right)=0
\]

\[
\boldsymbol{e}_{\theta_{1}}\cdot\boldsymbol{e}_{\chi}=\left(a^{2}\,cosh(\chi)cos(\theta_{1})sin^{2}(\theta_{2})sinh(\chi)sin(\theta_{1})-a^{2}\,cosh(\chi)sin(\theta_{1})sin^{2}(\theta_{2})sinh(\chi)cos(\theta_{1})\right)=0
\]

\[
\boldsymbol{e}_{\theta_{2}}\cdot\boldsymbol{e}_{\chi}=\left(a^{2}\,cosh(\chi)sin^{2}(\theta_{1})cos(\theta_{2})\,sinh(\chi)sin(\theta_{2})+a^{2}\,cosh(\chi)cos^{2}(\theta_{1})cos(\theta_{2})sinh(\chi)sin(\theta_{2}) \right.
\]

\[ \left.-a^{2}\,sinh(\chi)sin(\theta_{2})cosh(\chi)cos(\theta_{2})\right)\]

\[
=a^{2}\,cosh(\chi)cos(\theta_{2})\,sinh(\chi)sin(\theta_{2})\left(sin^{2}(\theta_{1})+cos^{2}(\theta_{1})\right)-a^{2}\,sinh(\chi)sin(\theta_{2})cosh(\chi)cos(\theta_{2})
\]

\[
=a^{2}\,cosh(\chi)cos(\theta_{2})\,sinh(\chi)sin(\theta_{2})-a^{2}\,sinh(\chi)sin(\theta_{2})cosh(\chi)cos(\theta_{2})=0
\]

Verificada a ortogonalidade entre os vetores da base, pode-se dizer
que o sistema de coordenadas � ortogonal.

\section*{Verificando Ortogonalidade do Hiperbol�ide 3-dim (3-Hiperboloide?)
de 2 Folhas}

Verificar a ortogonalidade do seguinte sistema de coordenadas

\begin{eqnarray*}
x_{3} & = & a\cosh\chi\cos\theta_{2}\\
x_{2} & = & a\sinh\chi\sin\theta_{2}\cos\theta_{1}\\
x_{1} & = & a\sinh\chi\sin\theta_{2}\sin\theta_{1}\\
0 & \leq & \chi<\infty,\ 0\leq\theta_{2}\leq\pi,\ 0\leq\theta_{1}<2\pi
\end{eqnarray*}

Define-se

\[
\boldsymbol{r=}(a\sinh\chi\sin\theta_{2}\sin\theta_{1})\boldsymbol{\widehat{i}}+(a\sinh\chi\sin\theta_{2}\cos\theta_{1})\boldsymbol{\widehat{j}}+(a\cosh\chi\cos\theta_{2})\boldsymbol{\widehat{k}}
\]

Tem-se

\[
\boldsymbol{e}_{\theta_{1}}=\frac{\partial\boldsymbol{r}}{\partial\theta_{1}}=\left(\frac{\partial}{\partial\theta_{1}}(a\sinh\chi\sin\theta_{2}\sin\theta_{1})\boldsymbol{\widehat{i}}+\frac{\partial}{\partial\theta_{1}}(a\sinh\chi\sin\theta_{2}\cos\theta_{1})\boldsymbol{\widehat{j}}+\frac{\partial}{\partial\theta_{1}}(a\cosh\chi\cos\theta_{2})\boldsymbol{\widehat{k}}\right)
\]

\[
=\left(a\,sinh(\chi)cos(\theta_{1})sin(\theta_{2})\boldsymbol{\widehat{i}}-a\,sinh(\chi)sin(\theta_{1})sin(\theta_{2})\boldsymbol{\widehat{j}}+0\boldsymbol{\widehat{k}}\right)
\]

\[
\boldsymbol{e}_{\theta_{2}}=\frac{\partial\boldsymbol{r}}{\partial\theta_{2}}=\left(\frac{\partial}{\partial\theta_{2}}(a\sinh\chi\sin\theta_{2}\sin\theta_{1})\boldsymbol{\widehat{i}}+\frac{\partial}{\partial\theta_{2}}(a\sinh\chi\sin\theta_{2}\cos\theta_{1})\boldsymbol{\widehat{j}}+\frac{\partial}{\partial\theta_{2}}(a\cosh\chi\cos\theta_{2})\boldsymbol{\widehat{k}}\right)
\]

\[
=\left(a\,sinh(\chi)sin(\theta_{1})cos(\theta_{2})\boldsymbol{\widehat{i}}+a\,sinh(\chi)cos(\theta_{1})cos(\theta_{2})\boldsymbol{\widehat{j}}-a\,cosh(\chi)sin(\theta_{2})\boldsymbol{\widehat{k}}\right)
\]

\[
\boldsymbol{e}_{\chi}=\frac{\partial\boldsymbol{r}}{\partial\chi}=\left(\frac{\partial}{\partial\chi}(a\sinh\chi\sin\theta_{2}\sin\theta_{1})\boldsymbol{\widehat{i}}+\frac{\partial}{\partial\chi}(a\sinh\chi\sin\theta_{2}\cos\theta_{1})\boldsymbol{\widehat{j}}+\frac{\partial}{\partial\chi}(a\cosh\chi\cos\theta_{2})\boldsymbol{\widehat{k}}\right)
\]

\[
=\left(a\,cosh(\chi)sin(\theta_{1})sin(\theta_{2})\boldsymbol{\widehat{i}}+a\,cosh(\chi)cos(\theta_{1})sin(\theta_{2})\boldsymbol{\widehat{j}}+a\,sinh(\chi)cos(\theta_{2})\boldsymbol{\widehat{k}}\right)
\]

Agora basta verificar a ortogonalidade dois a dois entre estes vetores.

\[
\boldsymbol{e}_{\theta_{1}}\cdot\boldsymbol{e}_{\theta_{2}}=\left(a^{2}\,sinh^{2}(\chi)cos(\theta_{1})sin(\theta_{2})sin(\theta_{1})cos(\theta_{2})-a^{2}\,sinh^{2}(\chi)sin(\theta_{1})sin(\theta_{2})cos(\theta_{1})cos(\theta_{2})\right)=0
\]

\[
\boldsymbol{e}_{\theta_{1}}\cdot\boldsymbol{e}_{\chi}=\left(a^{2}\,cosh(\chi)cos(\theta_{1})sin^{2}(\theta_{2})sinh(\chi)sin(\theta_{1})-a^{2}\,cosh(\chi)sin(\theta_{1})sin^{2}(\theta_{2})sinh(\chi)cos(\theta_{1})\right)=0
\]

\[
\boldsymbol{e}_{\theta_{2}}\cdot\boldsymbol{e}_{\chi}=\left(a^{2}\,cosh(\chi)sin^{2}(\theta_{1})cos(\theta_{2})\,sinh(\chi)sin(\theta_{2})+a^{2}\,cosh(\chi)cos^{2}(\theta_{1})cos(\theta_{2})sinh(\chi)sin(\theta_{2}) \right.
\]

\[ \left.-a^{2}\,sinh(\chi)sin(\theta_{2})cosh(\chi)cos(\theta_{2})\right) \]

\[
=a^{2}\,cosh(\chi)cos(\theta_{2})\,sinh(\chi)sin(\theta_{2})\left(sin^{2}(\theta_{1})+cos^{2}(\theta_{1})\right)-a^{2}\,sinh(\chi)sin(\theta_{2})cosh(\chi)cos(\theta_{2})
\]

\[
=a^{2}\,cosh(\chi)cos(\theta_{2})\,sinh(\chi)sin(\theta_{2})-a^{2}\,sinh(\chi)sin(\theta_{2})cosh(\chi)cos(\theta_{2})=0
\]
\\
\\
Verificada a ortogonalidade entre os vetores da base, pode-se dizer
que o sistema de coordenadas � ortogonal.

\section*{Verificando Ortogonalidade do Hiperbol�ide 4-dim (4-Hiperboloide?)
de 1 Folha}

Verificar a ortogonalidade do seguinte sistema de coordenadas

\begin{eqnarray*}
x_{4} & = & a\sinh\chi\cos\theta_{3}\\
x_{3} & = & a\cosh\chi\sin\theta_{3}\cos\theta_{2}\\
x_{1} & = & a\cosh\chi\sin\theta_{3}\sin\theta_{2}\cos\theta_{1}\\
x_{2} & = & a\cosh\chi\sin\theta_{3}\sin\theta_{2}\sin\theta_{1}\\
0 & \leq & \chi<\infty,\ 0\leq\theta_{3},\theta_{2}\leq\pi,\ 0\leq\theta_{1}<2\pi
\end{eqnarray*}

Define-se

\[
\boldsymbol{r=}\left(a\cosh\chi\sin\theta_{3}\sin\theta_{2}\cos\theta_{1},\,a\cosh\chi\sin\theta_{3}\sin\theta_{2}\sin\theta_{1},\,a\cosh\chi\sin\theta_{3}\cos\theta_{2},\,a\sinh\chi\cos\theta_{3}\right)
\]

Tem-se

\[
\boldsymbol{e}_{\theta_{1}}=\frac{\partial\boldsymbol{r}}{\partial\theta_{1}}=\left(\frac{\partial}{\partial\theta_{1}}(a\cosh\chi\sin\theta_{3}\sin\theta_{2}\cos\theta_{1}),\,\frac{\partial}{\partial\theta_{1}}(a\cosh\chi\sin\theta_{3}\sin\theta_{2}\sin\theta_{1}),\,\frac{\partial}{\partial\theta_{1}}(a\cosh\chi\sin\theta_{3}\cos\theta_{2}) \right.
\]
\[ \left.,\,\frac{\partial}{\partial\theta_{1}}(a\sinh\chi\cos\theta_{3})\right) \]
\[
=\left(-a\cosh\chi\sin\theta_{3}\sin\theta_{2}sin\theta_{1},a\cosh\chi\sin\theta_{3}\sin\theta_{2}cos\theta_{1},\,0,\,0\right)
\]

\[
\boldsymbol{e}_{\theta_{2}}=\frac{\partial\boldsymbol{r}}{\partial\theta_{2}}=\left(\frac{\partial}{\partial\theta_{2}}(a\cosh\chi\sin\theta_{3}\sin\theta_{2}\cos\theta_{1}),\,\frac{\partial}{\partial\theta_{2}}(a\cosh\chi\sin\theta_{3}\sin\theta_{2}\sin\theta_{1}),\,\frac{\partial}{\partial\theta_{2}}(a\cosh\chi\sin\theta_{3}\cos\theta_{2}),\, \right.
\]
\[ \left. \frac{\partial}{\partial\theta_{2}}(a\sinh\chi\cos\theta_{3})\right) \]
\[
=\left(a\cosh\chi\sin\theta_{3}cos\theta_{2}\cos\theta_{1},\,a\cosh\chi\sin\theta_{3}cos\theta_{2}\sin\theta_{1},\,-a\cosh\chi\sin\theta_{3}sin\theta_{2},0\right)
\]

\[
\boldsymbol{e}_{\theta_{3}}=\frac{\partial\boldsymbol{r}}{\partial\theta_{3}}=\left(\frac{\partial}{\partial\theta_{3}}(a\cosh\chi\sin\theta_{3}\sin\theta_{2}\cos\theta_{1}),\,\frac{\partial}{\partial\theta_{3}}(a\cosh\chi\sin\theta_{3}\sin\theta_{2}\sin\theta_{1}),\,\frac{\partial}{\partial\theta_{3}}(a\cosh\chi\sin\theta_{3}\cos\theta_{2}),\, \right.\]
\[ \left.\frac{\partial}{\partial\theta_{3}}(a\sinh\chi\cos\theta_{3})\right)
\ \]
\[
=\left(a\cosh\chi cos\theta_{3}\sin\theta_{2}\cos\theta_{1},\,a\cosh\chi cos\theta_{3}\sin\theta_{2}\sin\theta_{1},a\cosh\chi cos\theta_{3}\cos\theta_{2},\,-a\sinh\chi sin\theta_{3}\right)
\]

\[
\boldsymbol{e}_{\chi}=\frac{\partial\boldsymbol{r}}{\partial\chi}=\left(\frac{\partial}{\partial\chi}(a\cosh\chi\sin\theta_{3}\sin\theta_{2}\cos\theta_{1}),\,\frac{\partial}{\partial\chi}(a\cosh\chi\sin\theta_{3}\sin\theta_{2}\sin\theta_{1}),\,\frac{\partial}{\partial\chi}(a\cosh\chi\sin\theta_{3}\cos\theta_{2}),\right.
\]

\[ \left.\,\frac{\partial}{\partial\chi}(a\sinh\chi\cos\theta_{3})\right) \]

\[
=\left(a\,sinh\chi\sin\theta_{3}\sin\theta_{2}\cos\theta_{1},\,a\,sinh\chi\sin\theta_{3}\sin\theta_{2}\sin\theta_{1},\,a\,sinh\chi\sin\theta_{3}\cos\theta_{2},\,a\,cosh\chi\cos\theta_{3}\right)
\]
\\
Agora basta verificar a ortogonalidade dois a dois entre estes vetores.

\[
\boldsymbol{e}_{\theta_{1}}\cdot\boldsymbol{e}_{\theta_{2}}=\left(-a^{2}\cosh^{2}\chi\sin^{2}\theta_{3}\sin\theta_{2}sin\theta_{1}\,cos\theta_{2}\cos\theta_{1}+a^{2}\cosh^{2}\chi\sin^{2}\theta_{3}\sin\theta_{2}cos\theta_{1}cos\theta_{2}\sin\theta_{1}\right)=0
\]

\[
\boldsymbol{e}_{\theta_{1}}\cdot\boldsymbol{e}_{\theta_{3}}=\left(-a^{2}\cosh^{2}\chi\sin\theta_{3}\sin^{2}\theta_{2}sin\theta_{1}cos\theta_{3}\cos\theta_{1}+a^{2}\cosh^{2}\chi\sin\theta_{3}\sin^{2}\theta_{2}cos\theta_{1}cos\theta_{3}\sin\theta_{1}\right)=0
\]

\[
\boldsymbol{e}_{\theta_{1}}\cdot\boldsymbol{e}_{\chi}=\left(-a^{2}\cosh\chi\sin^{2}\theta_{3}\sin^{2}\theta_{2}sin\theta_{1}sinh\chi\cos\theta_{1}+a^{2}\cosh\chi\sin^{2}\theta_{3}\sin^{2}\theta_{2}cos\theta_{1}\,sinh\chi\sin\theta_{1}\right)=0
\]

\[
\boldsymbol{e}_{\theta_{2}}\cdot\boldsymbol{e}_{\theta_{3}}=\left(a^{2}\cosh^{2}\chi\sin\theta_{3}cos\theta_{2}\cos^{2}\theta_{1}cos\theta_{3}\sin\theta_{2}+a^{2}\cosh^{2}\chi\sin\theta_{3}cos\theta_{2}\sin^{2}\theta_{1}cos\theta_{3}\sin\theta_{2} \right.
\]

\[ \left.-a^{2}\cosh^{2}\chi\sin\theta_{3}sin\theta_{2}cos\theta_{3}\cos\theta_{2}\right) \]

\[
=\left(a^{2}\cosh^{2}\chi\sin\theta_{3}cos\theta_{2}cos\theta_{3}\sin\theta_{2}-a^{2}\cosh^{2}\chi\sin\theta_{3}sin\theta_{2}cos\theta_{3}\cos\theta_{2}\right)=0
\]

\[
\boldsymbol{e}_{\theta_{2}}\cdot\boldsymbol{e}_{\chi}=\left(a^{2}\cosh\chi\sin^{2}\theta_{3}cos\theta_{2}\cos^{2}\theta_{1}sinh\chi\sin\theta_{2}+\,a^{2}\cosh\chi\sin^{2}\theta_{3}cos\theta_{2}\sin^{2}\theta_{1}\,sinh\chi\sin\theta_{2} \right.
\]

\[ \left.-a^{2}\cosh\chi\sin^{2}\theta_{3}sin\theta_{2}\,sinh\chi\cos\theta_{2}\right) \]

\[
=\left(a^{2}\cosh\chi\sin^{2}\theta_{3}cos\theta_{2}sinh\chi\sin\theta_{2}-a^{2}\cosh\chi\sin^{2}\theta_{3}sin\theta_{2}\,sinh\chi\cos\theta_{2}\right)=0
\]

\[
\boldsymbol{e}_{\theta_{3}}\cdot\boldsymbol{e}_{\chi}=(a^{2}\cosh\chi cos\theta_{3}\sin^{2}\theta_{2}\cos^{2}\theta_{1}sinh\chi\sin\theta_{3}+a^{2}\cosh\chi cos\theta_{3}\sin^{2}\theta_{2}\sin^{2}\theta_{1}\,sinh\chi\sin\theta_{3}+
\]
\[a^{2}\cosh\chi cos\theta_{3}\cos^{2}\theta_{2}\,sinh\chi\sin\theta_{3}-a^{2}\sinh\chi sin\theta_{3}\,cosh\chi\cos\theta_{3})\]


\[
=a^{2}\cosh\chi cos\theta_{3}\sin^{2}\theta_{2}sinh\chi\sin\theta_{3}+a^{2}\cosh\chi cos\theta_{3}\cos^{2}\theta_{2}\,sinh\chi\sin\theta_{3}-a^{2}\sinh\chi sin\theta_{3}\,cosh\chi\cos\theta_{3}
\]

\[
=a^{2}\cosh\chi cos\theta_{3}sinh\chi\sin\theta_{3}-a^{2}\sinh\chi sin\theta_{3}\,cosh\chi\cos\theta_{3}=0
\]
\\
Verificada a ortogonalidade entre os vetores da base, pode-se dizer
que o sistema de coordenadas � ortogonal.

\section*{Verificando Ortogonalidade do Hiperbol�ide 4-dim (4-Hiperboloide?)
de 2 Folhas}

Verificar a ortogonalidade do seguinte sistema de coordenadas

\begin{eqnarray*}
x_{4} & = & a\cosh\chi\cos\theta_{3}\\
x_{3} & = & a\sinh\chi\sin\theta_{3}\cos\theta_{2}\\
x_{2} & = & a\sinh\chi\sin\theta_{3}\sin\theta_{2}\cos\theta_{1}\\
x_{1} & = & a\sinh\chi\sin\theta_{3}\sin\theta_{2}\sin\theta_{1}\\
0 & \leq & \chi<\infty,\ 0\leq\theta_{3},\theta_{2}\leq\pi,\ 0\leq\theta_{1}<2\pi
\end{eqnarray*}

Define-se

\[
\boldsymbol{r=}\left(a\sinh\chi\sin\theta_{3}\sin\theta_{2}\sin\theta_{1},\,a\sinh\chi\sin\theta_{3}\sin\theta_{2}\cos\theta_{1},\,a\sinh\chi\sin\theta_{3}\cos\theta_{2},\,a\cosh\chi\cos\theta_{3}\right)
\]

Tem-se

\[
\boldsymbol{e}_{\theta_{1}}=\frac{\partial\boldsymbol{r}}{\partial\theta_{1}}=\left(\frac{\partial}{\partial\theta_{1}}(a\sinh\chi\sin\theta_{3}\sin\theta_{2}\sin\theta_{1}),\,\frac{\partial}{\partial\theta_{1}}(a\sinh\chi\sin\theta_{3}\sin\theta_{2}\cos\theta_{1}),\,\frac{\partial}{\partial\theta_{1}}(\,a\sinh\chi\sin\theta_{3}\cos\theta_{2}), \right.
\]

\[ \left.\,\frac{\partial}{\partial\theta_{1}}(a\cosh\chi\cos\theta_{3})\right) \]

\[
=\left(a\,sinh\chi\sin\theta_{3}\sin\theta_{2}cos\theta_{1},-a\sinh\chi\sin\theta_{3}\sin\theta_{2}sin\theta_{1},\,0,\,0\right)
\]

\[
\boldsymbol{e}_{\theta_{2}}=\frac{\partial\boldsymbol{r}}{\partial\theta_{2}}=\left(\frac{\partial}{\partial\theta_{2}}(a\sinh\chi\sin\theta_{3}\sin\theta_{2}\sin\theta_{1}),\,\frac{\partial}{\partial\theta_{2}}(a\sinh\chi\sin\theta_{3}\sin\theta_{2}\cos\theta_{1}),\,\frac{\partial}{\partial\theta_{2}}(\,a\sinh\chi\sin\theta_{3}\cos\theta_{2}), \right.
\]

\[ \left.\,\frac{\partial}{\partial\theta_{2}}(a\cosh\chi\cos\theta_{3})\right) \]

\[
=\left(a\sinh\chi\sin\theta_{3}cos\theta_{2}\sin\theta_{1},\,a\sinh\chi\sin\theta_{3}cos\theta_{2}\cos\theta_{1},\,-a\sinh\chi\sin\theta_{3}sin\theta_{2},0\right)
\]

\[
\boldsymbol{e}_{\theta_{3}}=\frac{\partial\boldsymbol{r}}{\partial\theta_{3}}=\left(\frac{\partial}{\partial\theta_{3}}(a\sinh\chi\sin\theta_{3}\sin\theta_{2}\sin\theta_{1}),\,\frac{\partial}{\partial\theta_{3}}(a\sinh\chi\sin\theta_{3}\sin\theta_{2}\cos\theta_{1}),\,\frac{\partial}{\partial\theta_{3}}(\,a\sinh\chi\sin\theta_{3}\cos\theta_{2}), \right.
\]

\[ \left.\,\frac{\partial}{\partial\theta_{3}}(a\cosh\chi\cos\theta_{3})\right) \]

\[
=\left(a\sinh\chi cos\theta_{3}\sin\theta_{2}\sin\theta_{1},\,a\sinh\chi cos\theta_{3}\sin\theta_{2}\cos\theta_{1},a\sinh\chi cos\theta_{3}\cos\theta_{2},\,-a\cosh\chi sin\theta_{3}\right)
\]

\[
\boldsymbol{e}_{\chi}=\frac{\partial\boldsymbol{r}}{\partial\chi}=\left(\frac{\partial}{\partial\chi}(a\sinh\chi\sin\theta_{3}\sin\theta_{2}\sin\theta_{1}),\,\frac{\partial}{\partial\chi}(a\sinh\chi\sin\theta_{3}\sin\theta_{2}\cos\theta_{1}),\,\frac{\partial}{\partial\chi}(\,a\sinh\chi\sin\theta_{3}\cos\theta_{2}), \right.
\]

\[ \left.\,\frac{\partial}{\partial\chi}(a\cosh\chi\cos\theta_{3})\right) \]

\[
=\left(a\,cosh\chi\sin\theta_{3}\sin\theta_{2}\sin\theta_{1},a\,cosh\chi\sin\theta_{3}\sin\theta_{2}\cos\theta_{1},\,a\,cosh\chi\sin\theta_{3}\cos\theta_{2},\,a\,sinh\chi\cos\theta_{3}\right)
\]
\\
Agora basta verificar a ortogonalidade dois a dois entre estes vetores.

\[
\boldsymbol{e}_{\theta_{1}}\cdot\boldsymbol{e}_{\theta_{2}}=\left(a^{2}\,sinh^{2}\chi\sin^{2}\theta_{3}\sin\theta_{2}cos\theta_{1}cos\theta_{2}\sin\theta_{1}-a^{2}\sinh^{2}\chi\sin^{2}\theta_{3}\sin\theta_{2}sin\theta_{1}\,cos\theta_{2}\cos\theta_{1}\right)=0
\]

\[
\boldsymbol{e}_{\theta_{1}}\cdot\boldsymbol{e}_{\theta_{3}}=\left(a^{2}\,sinh^{2}\chi\sin\theta_{3}\sin^{2}\theta_{2}cos\theta_{1}cos\theta_{3}\sin\theta_{1}-a^{2}\sinh^{2}\chi\sin\theta_{3}\sin^{2}\theta_{2}sin\theta_{1}cos\theta_{3}\cos\theta_{1}\right)=0
\]

\[
\boldsymbol{e}_{\theta_{1}}\cdot\boldsymbol{e}_{\chi}=\left(a^{2}\,sinh\chi\sin^{2}\theta_{3}\sin^{2}\theta_{2}cos\theta_{1}cosh\chi\sin\theta_{1}-a^{2}\sinh\chi\sin^{2}\theta_{3}\sin^{2}\theta_{2}sin\theta_{1}cosh\chi\cos\theta_{1}\right)=0
\]

\[
\boldsymbol{e}_{\theta_{2}}\cdot\boldsymbol{e}_{\theta_{3}}=\left(a^{2}\sinh^{2}\chi\sin\theta_{3}cos\theta_{2}\sin^{2}\theta_{1}cos\theta_{3}\sin\theta_{2}+a^{2}\sinh^{2}\chi\sin\theta_{3}cos\theta_{2}\cos^{2}\theta_{1}cos\theta_{3}\sin\theta_{2} \right.
\]

\[\left.-a^{2}\sinh^{2}\chi\sin\theta_{3}sin\theta_{2}cos\theta_{3}\cos\theta_{2}\right)\]

\[
=\left(a^{2}\sinh^{2}\chi\sin\theta_{3}cos\theta_{2}cos\theta_{3}\sin\theta_{2}-a^{2}\sinh^{2}\chi\sin\theta_{3}sin\theta_{2}cos\theta_{3}\cos\theta_{2}\right)=0
\]

\[
\boldsymbol{e}_{\theta_{2}}\cdot\boldsymbol{e}_{\chi}=\left(a^{2}\sinh\chi\sin^{2}\theta_{3}cos\theta_{2}\sin^{2}\theta_{1}cosh\chi\sin\theta_{2}+a^{2}\sinh\chi\sin^{2}\theta_{3}cos\theta_{2}\cos^{2}\theta_{1}cosh\chi\sin\theta_{2}\right.
\]

\[\left. -a^{2}\sinh\chi\sin^{2}\theta_{3}sin\theta_{2}cosh\chi\cos\theta_{2}\right)\]

\[
=\left(a^{2}\sinh\chi\sin^{2}\theta_{3}cos\theta_{2}cosh\chi\sin\theta_{2}-a^{2}\sinh\chi\sin^{2}\theta_{3}sin\theta_{2}cosh\chi\cos\theta_{2}\right)=0
\]

\[
\boldsymbol{e}_{\theta_{3}}\cdot\boldsymbol{e}_{\chi}=(a^{2}\sinh\chi cos\theta_{3}\sin^{2}\theta_{2}\sin^{2}\theta_{1}cosh\chi\sin\theta_{3}+a^{2}\sinh\chi cos\theta_{3}\sin^{2}\theta_{2}\cos^{2}\theta_{1}cosh\chi\sin\theta_{3}
\]

\[
+a^{2}\sinh\chi cos\theta_{3}\cos^{2}\theta_{2}cosh\chi\sin\theta_{3}-a^{2}\cosh\chi sin\theta_{3}sinh\chi\cos\theta_{3})
\]

\[
=a^{2}\sinh\chi cos\theta_{3}\sin^{2}\theta_{2}cosh\chi\sin\theta_{3}+a^{2}\sinh\chi cos\theta_{3}\cos^{2}\theta_{2}cosh\chi\sin\theta_{3}-a^{2}\cosh\chi sin\theta_{3}sinh\chi\cos\theta_{3}
\]

\[
=a^{2}\sinh\chi cos\theta_{3}cosh\chi\sin\theta_{3}-a^{2}\cosh\chi sin\theta_{3}sinh\chi\cos\theta_{3}=0
\]
\\
Verificada a ortogonalidade entre os vetores da base, pode-se dizer
que o sistema de coordenadas � ortogonal.

\section{Conclus�o}

Mostrou-se a ortogonalidade das 4 parametriza��es estudadas. Dada
essa caracter�stica dos sistemas de coordenadas vale, de acordo com
Riley(2006), a seguinte propriedade:

\[
g_{ij}=\begin{cases}
h_{i}^{2} & ,i=j\\
0 & ,i\neq j
\end{cases}
\]

\end{document}
