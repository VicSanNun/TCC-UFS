
\documentclass[a4paper,12pt]{article}
%%%%%%%%%%%%%%%%%%%%%%%%%%%%%%%%%%%%%%%%%%%%%%%%%%%%%%%%%%%%%%%%%%%%%%%%%%%%%%%%%%%%%%%%%%%%%%%%%%%%%%%%%%%%%%%%%%%%%%%%%%%%%%%%%%%%%%%%%%%%%%%%%%%%%%%%%%%%%%%%%%%%%%%%%%%%%%%%%%%%%%%%%%%%%%%%%%%%%%%%%%%%%%%%%%%%%%%%%%%%%%%%%%%%%%%%%%%%%%%%%%%%%%%%%%%%
\usepackage{amssymb}
\usepackage{amsfonts}
\usepackage{graphicx}
\usepackage{amsmath}

\setcounter{MaxMatrixCols}{10}
%TCIDATA{OutputFilter=LATEX.DLL}
%TCIDATA{Version=5.50.0.2953}
%TCIDATA{<META NAME="SaveForMode" CONTENT="1">}
%TCIDATA{BibliographyScheme=Manual}
%TCIDATA{Created=Sat Oct 14 01:01:49 2006}
%TCIDATA{LastRevised=Wednesday, July 07, 2021 09:30:17}
%TCIDATA{<META NAME="GraphicsSave" CONTENT="32">}
%TCIDATA{<META NAME="DocumentShell" CONTENT="Journal Articles\Standard LaTeX Article">}
%TCIDATA{CSTFile=LaTeX article (bright).cst}

\textheight=26.5cm
\textwidth=18cm
\oddsidemargin=-1.0cm
\evensidemargin=-1.0cm
\topmargin=-2.5cm
\def\baselinestretch{1.6}

\begin{document}


;

\section{Spheroidal coordinates em N+1 dimens\~{o}es}

;

vamos combinar de nota\c{c}\~{o}es. usam essas mesmas!

$a$ \'{e} uma constante; deixamos por enquanto, pode ser \'{u}til no futuro

prolate spheroidal coordinates - two-sheet hypeboloids%
\begin{eqnarray*}
x &=&a\sinh v\sin \theta \cos \varphi  \\
y &=&a\sinh v\sin \theta \sin \varphi  \\
z &=&a\cosh v\cos \theta  \\
0 &\leq &v<\infty ,\ 0\leq \theta \leq \pi ,\ 0\leq \varphi <2\pi 
\end{eqnarray*}%
\begin{eqnarray*}
x &=&a\sinh v\sin \theta \cos \varphi  \\
y &=&a\sinh v\sin \theta \sin \varphi  \\
z &=&a\cosh v\cos \theta 
\end{eqnarray*}%
\begin{equation*}
0\leq v<\infty ,\ 0\leq \theta \leq \pi ,\ 0\leq \varphi <2\pi 
\end{equation*}%
;

oblate spheroidal coordinates - one-sheet hypeboloids - o nosso caso atual%
\begin{eqnarray}
x &=&a\cosh v\cos \theta \cos \varphi   \label{00-osc} \\
y &=&a\cosh v\cos \theta \sin \varphi   \notag \\
z &=&a\sinh v\sin \theta   \notag \\
0 &\leq &v<\infty ,\ -\pi /2\leq \theta \leq \pi /2,\ 0\leq \varphi <2\pi  
\notag
\end{eqnarray}%
temos o sistema de hiperboloides:%
\begin{equation*}
\frac{x^{2}+y^{2}}{a^{2}\cos ^{2}\theta }-\frac{z^{2}}{a^{2}\sin ^{2}\theta }%
=1
\end{equation*}%
;

depois passa para coordenadas%
\begin{equation*}
\xi =\sin \theta ,\ \zeta =\sinh v,\ \varphi
\end{equation*}%
$\frac{{}}{{}}$

\textit{Uma observa\c{c}\~{a}o.}

\{para prepara\c{c}\~{a}o para o caso de altas dimens\~{o}es e ter uma
analogia com coordenadas esf\'{e}ricas e com a parametriza\c{c}\~{a}o de
two-sheet hypeboloid \'{e} melhor usar a parametriza\c{c}\~{a}o um pouco
diferente:%
\begin{eqnarray}
x &=&a\cosh v\sin \theta \cos \varphi   \label{00-osc2} \\
y &=&a\cosh v\sin \theta \sin \varphi   \notag \\
z &=&a\sinh v\cos \theta   \notag \\
0 &\leq &v<\infty ,\ 0\leq \theta \leq \pi ,\ 0\leq \varphi <2\pi   \notag
\end{eqnarray}%
nesse caso as express\~{o}es para coeficientes m\'{e}tricos e para
Laplaciano mudam. mas para dim 4 isso ja pode ser \'{u}til.

\}

$\frac{{}}{{}}$

por enquanto na sua descri\c{c}\~{a}o vamos usar a parametriza\c{c}\~{a}o da
forma (\ref{00-osc}) como est\'{a} no site

na descri\c{c}\~{a}o deve obter a m\'{e}trica na forma expl\'{\i}cita para
usa-lo na t\'{e}cnica do operador de Laplace-Beltrami e depois na eq de
Dirac. para obter a metrica usa o material do Aleff.

\{qual \'{e} arq do relatorio do Aleff vc tem? pode me enviar. vou fazer
referecias na formulas desse arq.\}

para usar a metodologia descrita no rel do Aleff vamos usar a parametriza%
\c{c}\~{a}o da forma (\ref{00-osc2}) e a nota\c{c}\~{a}o de coord.s
Cartesianas seguinte:%
\begin{eqnarray*}
x_{3} &=&a\sinh \chi \cos \theta _{2} \\
x_{2} &=&a\cosh \chi \sin \theta _{2}\sin \theta _{1} \\
x_{1} &=&a\cosh \chi \sin \theta _{2}\cos \theta _{1}
\end{eqnarray*}%
escrevemos o sistema de 2-hiperboloides como:%
\begin{equation*}
\frac{1}{a^{2}\sin ^{2}\theta _{2}}\sum_{i=1}^{2}x_{i}^{2}-\frac{1}{%
a^{2}\cos ^{2}\theta _{2}}x_{N+1}^{2}=1
\end{equation*}%
quando passar para dimens\~{a}o 4 ser\'{a} mais f\'{a}cil generalizar:%
\begin{eqnarray*}
x_{4} &=&a\sinh \chi \cos \theta _{3} \\
x_{3} &=&a\cosh \chi \sin \theta _{3}\cos \theta _{2} \\
x_{2} &=&a\cosh \chi \sin \theta _{3}\sin \theta _{2}\sin \theta _{1} \\
x_{1} &=&a\cosh \chi \sin \theta _{3}\sin \theta _{2}\cos \theta _{1}
\end{eqnarray*}%
temos nesse caso temos o sistema de 3-hiperboloides:%
\begin{equation*}
\frac{1}{a^{2}\sin ^{2}\theta _{3}}\sum_{i=1}^{3}x_{i}^{2}-\frac{1}{%
a^{2}\cos ^{2}\theta _{3}}x_{4}^{2}=1
\end{equation*}%
na caso de dim $\left( N+1\right) $ temos o sistema de N-hiperboloides:%
\begin{equation*}
\frac{1}{a^{2}\sin ^{2}\theta _{N}}\sum_{i=1}^{N}x_{i}^{2}-\frac{1}{%
a^{2}\cos ^{2}\theta _{N}}x_{N+1}^{2}=1
\end{equation*}%
e as coordenadas s\~{a}o%
\begin{eqnarray*}
x_{N+1} &=&a\sinh \chi \cos \theta _{N} \\
x_{N} &=&a\cosh \chi \sin \theta _{N}\cos \theta _{N-1} \\
x_{N-1} &=&a\cosh \chi \sin \theta _{N}\sin \theta _{N-1}\cos \theta _{N-2}
\\
&&... \\
x_{3} &=&a\cosh \chi \sin \theta _{N}...\sin \theta _{3}\cos \theta _{2} \\
x_{1} &=&a\cosh \chi \sin \theta _{N}...\sin \theta _{3}\sin \theta _{2}\cos
\theta _{1} \\
x_{2} &=&a\cosh \chi \sin \theta _{N}...\sin \theta _{3}\sin \theta _{2}\sin
\theta _{1}
\end{eqnarray*}%
;

\end{document}
