%% LyX 2.3.6.1 created this file.  For more info, see http://www.lyx.org/.
%% Do not edit unless you really know what you are doing.
\documentclass[english]{article}
\usepackage[T1]{fontenc}
\usepackage[latin9]{inputenc}
\usepackage{amsmath}
\usepackage{babel}
\begin{document}
Encontrar Laplaciano nessas coordenadas

\begin{eqnarray*}
x_{4} & = & a\sinh\chi\cos\theta_{3}\\
x_{3} & = & a\cosh\chi\sin\theta_{3}\cos\theta_{2}\\
x_{2} & = & a\cosh\chi\sin\theta_{3}\sin\theta_{2}\cos\theta_{1}\\
x_{1} & = & a\cosh\chi\sin\theta_{3}\sin\theta_{2}\sin\theta_{1}\\
0 & \leq & \chi<\infty,\ 0\leq\theta_{3},\theta_{2}\leq\pi,\ 0\leq\theta_{1}<2\pi
\end{eqnarray*}

Utilizando o Operador de Laplace-Beltrami

\[
\Delta u=\frac{1}{\sqrt{\left\vert g\right\vert }}\frac{\partial}{\partial q_{i}}\left(\sqrt{\left\vert g\right\vert }g^{ij}\frac{\partial u}{\partial q_{j}}\right)
\]

De acordo com Arfken(2017), podemos determinar o tensor m�trico $g_{ij}$

\[
G=(g_{ij})=\underset{k}{\sum}\frac{\partial x_{k}}{\partial q_{_{^{i}}}}\frac{\partial x_{^{_{k}}}}{\partial q_{_{^{j}}}}
\]

Por�m, de acordo com Riley(2006), se o sistema de coordenadas for
ortogonal tem-se:

\[
g_{ij}=\begin{cases}
h_{i}^{2} & ,i=j\\
0 & ,i\neq j
\end{cases}
\]

\[
G=(g_{ij})=\left(\begin{array}{cccc}
h_{1}^{2} & 0 & 0 & 0\\
0 & h_{2}^{2} & 0 & 0\\
0 & 0 & h_{3}^{2} & 0\\
0 & 0 & 0 & h_{4}^{2}
\end{array}\right)
\]

\[
g^{ij}=\begin{cases}
1/h_{i}^{2} & ,i=j\\
0 & ,i\neq j
\end{cases}
\]

\[
G^{-1}=(g^{ij})=\left(\begin{array}{cccc}
\frac{1}{h_{1}^{2}} & 0 & 0 & 0\\
0 & \frac{1}{h_{2}^{2}} & 0 & 0\\
0 & 0 & \frac{1}{h_{3}^{2}} & 0\\
0 & 0 & 0 & \frac{1}{h_{4}^{2}}
\end{array}\right)
\]

E temos o determinante $g=det(g_{ij})=h_{1}^{2}h_{2}^{2}h_{3}^{2}h_{4}^{2}$

Determinando os Coeficientes M�tricos:
\begin{align*}
h_{1}^{2} & =h_{\theta_{1}}^{2}=a^{2}\sin^{2}\left(\theta_{2}\right)\sin^{2}\left(\theta_{3}\right)\cosh^{2}\left(\chi\right)
\end{align*}

\[
h_{2}^{2}=h_{\theta_{2}}^{2}=a^{2}\sin^{2}\left(\theta_{3}\right)\cosh^{2}\left(\chi\right)
\]

\[
h_{3}^{2}=h_{\theta_{3}}^{2}=-a^{2}[\sin^{2}\left(\theta_{3}\right)-\cosh^{2}\left(\chi\right)]
\]

\[
h_{4}^{2}=h_{\chi}^{2}=-a^{2}[\sin^{2}\left(\theta_{3}\right)-\cosh^{2}\left(\chi\right)]
\]

Ora $h_{3}=h_{4}$ ent�o $\sqrt{|g|}=h_{1}h_{2}h_{3}^{2}$

\[
\sqrt{|g|}=h_{1}h_{2}h_{3}^{2}=a^{4}sin(\theta_{2})sin^{2}(\theta_{3})cosh^{2}(\chi)[\cosh^{2}\left(\chi\right)-\sin^{2}\left(\theta_{3}\right)]
\]

Ent�o a matriz $\sqrt{|g|}g^{ij}$:

\[
\sqrt{|g|}g^{ij}=\left(\begin{array}{cccc}
\frac{h_{2}h_{3}^{2}}{h_{1}} & 0 & 0 & 0\\
0 & \frac{h_{1}h_{3}^{2}}{h_{2}} & 0 & 0\\
0 & 0 & h_{1}h_{2} & 0\\
0 & 0 & 0 & h_{1}h_{2}
\end{array}\right)
\]

Utilizando o Operador de Laplace-Beltrami:

\[
\Delta u=\frac{1}{\sqrt{\left\vert g\right\vert }}\left[\frac{\partial}{\partial\theta_{1}}\left(\sqrt{\left\vert g\right\vert }g^{11}\frac{\partial u}{\partial\theta_{1}}\right)+\frac{\partial}{\partial\theta_{2}}\left(\sqrt{\left\vert g\right\vert }g^{22}\frac{\partial u}{\partial\theta_{2}}\right)+\frac{\partial}{\partial\theta_{3}}\left(\sqrt{\left\vert g\right\vert }g^{33}\frac{\partial u}{\partial\theta_{3}}\right)+\frac{\partial}{\partial\chi}\left(\sqrt{\left\vert g\right\vert }g^{44}\frac{\partial u}{\partial\chi}\right)\right]
\]

Fazendo por partes:

\[
\frac{\partial}{\partial\theta_{1}}\left(\sqrt{\left\vert g\right\vert }g^{11}\frac{\partial u}{\partial\theta_{1}}\right)=\frac{\partial}{\partial\theta_{1}}\left(-\frac{a^{2}[\sin^{2}\left(\theta_{3}\right)-\cosh^{2}\left(\chi\right)]}{\sin\left(\theta_{2}\right)}\frac{\partial u}{\partial q_{1}}\right)=\left(-\frac{a^{2}[\sin^{2}\left(\theta_{3}\right)-\cosh^{2}\left(\chi\right)]}{\sin\left(\theta_{2}\right)}\right)\frac{\partial^{2}u}{\partial\theta_{1}^{2}}
\]

\[
\frac{\partial}{\partial\theta_{2}}\left(\sqrt{\left\vert g\right\vert }g^{22}\frac{\partial u}{\partial\theta_{2}}\right)=\frac{\partial}{\partial\theta_{2}}\left(-a^{2}sin(\theta_{2})[\sin^{2}\left(\theta_{3}\right)-\cosh^{2}\left(\chi\right)]\frac{\partial u}{\partial\theta_{2}}\right)
\]

\[
\frac{\partial}{\partial\theta_{3}}\left(\sqrt{\left\vert g\right\vert }g^{33}\frac{\partial u}{\partial\theta_{3}}\right)=\frac{\partial}{\partial\theta_{3}}\left(a^{2}\sin\left(\theta_{2}\right)\sin^{2}\left(\theta_{3}\right)\cosh^{2}\left(\chi\right)\frac{\partial u}{\partial\theta_{3}}\right)
\]

\[
\frac{\partial}{\partial\chi}\left(\sqrt{\left\vert g\right\vert }g^{44}\frac{\partial u}{\partial\chi}\right)=\frac{\partial}{\partial\chi}\left(a^{2}\sin\left(\theta_{2}\right)\sin^{2}\left(\theta_{3}\right)\cosh^{2}\left(\chi\right)\frac{\partial u}{\partial\chi}\right)
\]

Multiplicando $\frac{\partial}{\partial\theta_{1}}\left(\sqrt{\left\vert g\right\vert }g^{11}\frac{\partial u}{\partial\theta_{1}}\right)$
pelo fator $\frac{1}{\sqrt{\left\vert g\right\vert }}$, encontramos:

\[
\frac{1}{\sqrt{\left\vert g\right\vert }}\frac{\partial}{\partial\theta_{1}}\left(\sqrt{\left\vert g\right\vert }g^{11}\frac{\partial u}{\partial\theta_{1}}\right)=\left(-\frac{1}{a^{2}sin^{2}(\theta_{2})sin^{3}(\theta_{3})cosh^{2}(\chi)}\right)\frac{\partial^{2}u}{\partial\theta_{1}^{2}}
\]

Ent�o temos o Laplaciano:

\[
\Delta u=\frac{1}{a^{4}sin(\theta_{2})sin^{2}(\theta_{3})cosh^{2}(\chi)[\cosh^{2}\left(\chi\right)-\sin^{2}\left(\theta_{3}\right)]}[\frac{\partial}{\partial\theta_{2}}\left(-a^{2}sin(\theta_{2})[\sin^{2}\left(\theta_{3}\right)-\cosh^{2}\left(\chi\right)]\frac{\partial u}{\partial\theta_{2}}\right)
\]

\[
+\frac{\partial}{\partial\theta_{3}}\left(a^{2}\sin\left(\theta_{2}\right)\sin^{2}\left(\theta_{3}\right)\cosh^{2}\left(\chi\right)\frac{\partial u}{\partial\theta_{3}}\right)
\]

\[
+\frac{\partial}{\partial\chi}\left(a^{2}\sin\left(\theta_{2}\right)\sin^{2}\left(\theta_{3}\right)\cosh^{2}\left(\chi\right)\frac{\partial u}{\partial\chi}\right)]
\]

\[
-\left(\frac{1}{a^{2}sin^{2}(\theta_{2})sin^{3}(\theta_{3})cosh^{2}(\chi)}\right)\frac{\partial^{2}u}{\partial\theta_{1}^{2}}
\]

\end{document}
