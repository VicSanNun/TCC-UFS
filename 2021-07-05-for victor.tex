
\documentclass[a4paper,12pt]{article}
%%%%%%%%%%%%%%%%%%%%%%%%%%%%%%%%%%%%%%%%%%%%%%%%%%%%%%%%%%%%%%%%%%%%%%%%%%%%%%%%%%%%%%%%%%%%%%%%%%%%%%%%%%%%%%%%%%%%%%%%%%%%%%%%%%%%%%%%%%%%%%%%%%%%%%%%%%%%%%%%%%%%%%%%%%%%%%%%%%%%%%%%%%%%%%%%%%%%%%%%%%%%%%%%%%%%%%%%%%%%%%%%%%%%%%%%%%%%%%%%%%%%%%%%%%%%
\usepackage{amssymb}
\usepackage{amsfonts}
\usepackage{graphicx}
\usepackage{amsmath}

\setcounter{MaxMatrixCols}{10}
%TCIDATA{OutputFilter=LATEX.DLL}
%TCIDATA{Version=5.50.0.2953}
%TCIDATA{<META NAME="SaveForMode" CONTENT="1">}
%TCIDATA{BibliographyScheme=Manual}
%TCIDATA{Created=Sat Oct 14 01:01:49 2006}
%TCIDATA{LastRevised=Monday, July 05, 2021 11:40:42}
%TCIDATA{<META NAME="GraphicsSave" CONTENT="32">}
%TCIDATA{<META NAME="DocumentShell" CONTENT="Journal Articles\Standard LaTeX Article">}
%TCIDATA{CSTFile=LaTeX article (bright).cst}

\textheight=26.5cm
\textwidth=18cm
\oddsidemargin=-1.0cm
\evensidemargin=-1.0cm
\topmargin=-2.5cm
\def\baselinestretch{1.6}
\input{tcilatex}
\begin{document}


;

\section{Coordenadas hipeb\'{o}licas}

;

para o hiperboloide de uma folha

vamos combinar de nota\c{c}\~{o}es. usam essas mesmas! 

;

pode ser v\'{a}rias parametriza\c{c}\~{o}es. vamos iniciar dessa:

escrevemos as coordenadas hipeb\'{o}licas na forma 
\begin{eqnarray*}
x &=&r\cosh v\cos \theta  \\
y &=&r\cosh v\sin \theta  \\
z &=&r\sinh v
\end{eqnarray*}%
\begin{equation*}
0\leq r<\infty ,\ 0\leq \theta <2\pi ,\ -\infty <v<\infty 
\end{equation*}%
ent\~{a}o, as coordenadas descrevem hipeb\'{o}lides de uma folha%
\begin{equation*}
x^{2}+y^{2}-z^{2}=r^{2}
\end{equation*}%
ou%
\begin{equation*}
\frac{x^{2}}{r^{2}}+\frac{y^{2}}{r^{2}}-\frac{z^{2}}{r^{2}}=1
\end{equation*}%
o sistema de hiperboloides preencha todo o espa\c{c}o $\mathbb{R}^{3}$,
confirme a Fig. (em anexo)

;

entretanto, essa parametriza\c{c}\~{a}o n\~{a}o leva a ortogonalidade das
linhas de coordenadas. 

esse \'{e} o problema. pois n\~{a}o pode ser usadas diretamente as f\'{o}%
rmulas de div(grad) com coeficientes m\'{e}tricos para determina\c{c}\~{a}o
do Laplaciano.

temos que encontrar outra paramentriza\c{c}\~{a}o em que 

1. as linhas de coordenadas s\~{a}o ortogonais

2. o sistema de hiperboloides preencha todo o espa\c{c}o $\mathbb{R}^{3}$%
\begin{equation*}
\end{equation*}%
;

\end{document}
