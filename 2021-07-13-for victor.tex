
\documentclass[a4paper,12pt]{article}
%%%%%%%%%%%%%%%%%%%%%%%%%%%%%%%%%%%%%%%%%%%%%%%%%%%%%%%%%%%%%%%%%%%%%%%%%%%%%%%%%%%%%%%%%%%%%%%%%%%%%%%%%%%%%%%%%%%%%%%%%%%%%%%%%%%%%%%%%%%%%%%%%%%%%%%%%%%%%%%%%%%%%%%%%%%%%%%%%%%%%%%%%%%%%%%%%%%%%%%%%%%%%%%%%%%%%%%%%%%%%%%%%%%%%%%%%%%%%%%%%%%%%%%%%%%%
\usepackage{amssymb}
\usepackage{amsfonts}
\usepackage{graphicx}
\usepackage{amsmath}

\setcounter{MaxMatrixCols}{10}
%TCIDATA{OutputFilter=LATEX.DLL}
%TCIDATA{Version=5.50.0.2953}
%TCIDATA{<META NAME="SaveForMode" CONTENT="1">}
%TCIDATA{BibliographyScheme=Manual}
%TCIDATA{Created=Sat Oct 14 01:01:49 2006}
%TCIDATA{LastRevised=Tuesday, July 13, 2021 18:21:59}
%TCIDATA{<META NAME="GraphicsSave" CONTENT="32">}
%TCIDATA{<META NAME="DocumentShell" CONTENT="Journal Articles\Standard LaTeX Article">}
%TCIDATA{CSTFile=LaTeX article (bright).cst}

\textheight=26.5cm
\textwidth=18cm
\oddsidemargin=-1.0cm
\evensidemargin=-1.0cm
\topmargin=-2.5cm
\def\baselinestretch{1.6}

\begin{document}


;

\section{Laplaciano na parametriza\c{c}\~{a}o que descreve hiperboloide de 2
folhas}

;

determinar o Laplaciano na parametriza\c{c}\~{a}o que descreve hiperboloide
de 2 folhas

em 3 dim%
\begin{eqnarray*}
x_{3} &=&a\cosh \chi \cos \theta _{2} \\
x_{2} &=&a\sinh \chi \sin \theta _{2}\cos \theta _{1} \\
x_{1} &=&a\sinh \chi \sin \theta _{2}\sin \theta _{1} \\
0 &\leq &\chi <\infty ,\ 0\leq \theta _{2}\leq \pi ,\ 0\leq \theta _{1}<2\pi 
\end{eqnarray*}%
em 4 dim%
\begin{eqnarray*}
x_{4} &=&a\cosh \chi \cos \theta _{3} \\
x_{3} &=&a\sinh \chi \sin \theta _{3}\cos \theta _{2} \\
x_{2} &=&a\sinh \chi \sin \theta _{3}\sin \theta _{2}\cos \theta _{1} \\
x_{1} &=&a\sinh \chi \sin \theta _{3}\sin \theta _{2}\sin \theta _{1} \\
0 &\leq &\chi <\infty ,\ 0\leq \theta _{3},\theta _{2}\leq \pi ,\ 0\leq
\theta _{1}<2\pi 
\end{eqnarray*}%
ver a diferen\c{c}a com a parametriza\c{c}\~{a}o que usamos ja, que descreve
hiperboloide de 1 folha%
\begin{equation*}
\end{equation*}%
;

e ainda

- determinar curvatura do hiperboloide em 3d (2d-superficie) e 4d
(3d-hiper-superficie)

depois generalizamos para N-hiperboloide

;

como adjacente:

- determinar curvatura do elipsoide em 3d (2d-superficie) e 4d
(3d-hiper-superficie)

depois generalizamos para N-elipsoide

;

como ja aprendeu a t\'{e}cnica podemos aplica-la para determina\c{c}\~{a}o o
Laplaciano na pseudo-esfera (curvatura negativa constante)

no inicio em 3d (2d-superficie), depois em 4d (3d-hiper-superficie)

depois generalizamos para N-pseudo-esfera

\end{document}
